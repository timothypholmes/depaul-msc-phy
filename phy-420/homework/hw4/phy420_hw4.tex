\documentclass[11pt]{article}

\newcommand{\HWnum}{4} 
\newcommand{\StudName}{Timothy Holmes} % author
\newcommand{\CourseNum}{420}           % course number
\newcommand{\Subject}{PHY}

\usepackage{graphicx, amsmath, amssymb,fancyhdr}
\addtolength{\textwidth}{1.5in}
\addtolength{\oddsidemargin}{-2cm}
\addtolength{\evensidemargin}{-2cm}
\addtolength{\textheight}{1.6in}
\addtolength{\topmargin}{-0.7in}
\addtolength{\headsep}{-0.1in}
%\addtolength{\footskip}{-0.2in}
\pagestyle{fancy}
\cfoot{}
\lhead{\textbf{\Subject~\CourseNum~--- Homework~\HWnum}}
\rhead{\textbf{\StudName:~Page~\thepage}}

\addtolength{\parskip}{\baselineskip} % skips a line between paragraphs
\parindent 0in                        % no indent at start of paragraph

\newcommand{\dd}{\textrm{d}}
\usepackage{braket}
\usepackage{lipsum, babel}
\usepackage{blindtext}
\usepackage{graphicx}% Include figure files
\usepackage{dcolumn}% Align table columns on decimal point
\usepackage{bm}% bold math
\usepackage{listings}
\usepackage{listing}
\usepackage{supertabular}



\usepackage{color} %red, green, blue, yellow, cyan, magenta, black, white
\definecolor{mygreen}{RGB}{28,172,0} % color values Red, Green, Blue
\definecolor{mylilas}{RGB}{170,55,241}



\lstset{language=Python,%
    %basicstyle=\color{red},
    breaklines=true,%
    morekeywords={matlab2tikz},
    keywordstyle=\color{blue},%
    morekeywords=[2]{1}, keywordstyle=[2]{\color{black}},
    identifierstyle=\color{black},%
    stringstyle=\color{mylilas},
    commentstyle=\color{mygreen},%
    showstringspaces=false,%without this there will be a symbol in the places where there is a space
    numbers=left,%
    numberstyle={\tiny \color{black}},% size of the numbers
    numbersep=9pt, % this defines how far the numbers are from the text
    emph=[1]{for,end,break},emphstyle=[1]\color{red}, %some words to emphasise
    %emph=[2]{word1,word2}, emphstyle=[2]{style},    
}

\begin{document}
% -------------------------- BOD -------------------------- 

\title{Homework {\HWnum}}
\author{Timothy Holmes \\ \Subject ~ \CourseNum ~ Electrodynamics II}

\maketitle

\section*{Problem 1}

The wave equation for the vector potential $\vec{A}$ is gave by

$$
\nabla^{2} \vec{A} - \frac{1}{C^{2}} \frac{\partial^{2}\vec{A}}{\partial t^{2}} = - \mu_{0}\vec{J}
$$

\subsection*{(a)}

The Green function equation for the equation above can be expressed as

$$
\Bigg(\nabla^{2} - \frac{1}{c^{2}} \frac{\partial^{2}}{\partial t^{2}} \Bigg) G_{k}^{(\pm)} (\vec{x}, t; \vec{x}', t') = \mu_{0} \vec{J} \delta(\vec{x} - \vec{x}') \delta(t - t')  
$$

\subsection*{(b)}

$$
G^{(\pm)}(\vec{x}, t; \vec{x}', t') = \frac{1}{|\vec{x} - \vec{x}'|} \delta \Bigg[t' - t + \frac{|\vec{x} - \vec{x}'|}{c} \Bigg]
$$

\subsection*{(c)}

We first need

$$
\psi^{(\pm)}(\vec{x}, t) = \int \int G^{(\pm)}(\vec{x}, t; \vec{x}, t') f(\vec{x}, t) d^{3}x' dt' 
$$

then this can be expressed as

$$
\vec{A}(\vec{x}, t) = \int d^{3}x'\int  \frac{\vec{J}(\vec{x}', t')}{|\vec{x} - \vec{x}'|}\delta\Bigg(t' - \Bigg(t - \frac{|\vec{x} - \vec{x}'|}{c}\Bigg)\Bigg)
$$


\clearpage

\section*{Problem 2}

The vector potential $\vec{A}$ of an oscillating electric dipole is given by 

$$
\vec{A}(\vec{x}) = -\frac{i\mu_{0} \omega}{4\pi} \; \vec{p} \; \frac{e^{ikr}}{r}.
$$

To find its magnetic field we first have to write it as the curl of $\vec{A}$ such that

$$
\vec{H} = \frac{1}{\mu_{0}} \vec{\nabla} \times \vec{A}.
$$

$\vec{A}$ will only have the radial dependence, therefore, we only have to express this equation in terms of $x$. This is expressed as

$$
\vec{H} = \frac{1}{\mu_{0}}  \frac{\partial}{\partial x} \hat{n} \times \Bigg( -\frac{i\mu_{0}\omega}{4\pi} \; \vec{p} \; \frac{e^{ikx}}{x} \Bigg).
$$

Taking the partial derivative of $\vec{A}$ with respect to $x$ gives the result

\begin{align*}
    \vec{H} &= -\frac{i\omega}{4\pi} (\hat{n} \times \vec{p}) \Bigg(-\frac{e^{ikx}}{x^{2}} + \frac{ike^{ikx}}{x}\Bigg) \\
    \vec{H} &= \frac{ck^{2}}{4\pi} (\hat{n} \times \vec{p}) \frac{e^{ikx}}{x} \Bigg(1 - \frac{1}{ikx}\Bigg)
\end{align*}

Therefore, the final expression when $r = x$ is

$$
    \vec{H} = \frac{ck^{2}}{4\pi} (\hat{n} \times \vec{p}) \frac{e^{ikr}}{r} \Bigg(1 - \frac{1}{ikr}\Bigg)
$$

\clearpage

\section*{Problem 3}

The electric field of an oscillating electric dipole is given by

$$
\vec{E} = \frac{1}{4\pi \epsilon_{0}} \Bigg\{ k^{2} (\hat{n} \times \vec{p}) \times \hat{n} \frac{e^{ikr}}{r} + [3\hat{n}(\hat{n} \cdot \vec{p}) - \vec{p}] \Bigg(\frac{1}{r^{3}} - \frac{ik}{r^{2}}\Bigg )e^{ikr} \Bigg\}
$$

and in the prior question the magnetic field was found to be

$$
    \vec{H} = \frac{ck^{2}}{4\pi} (\hat{n} \times \vec{p}) \frac{e^{ikr}}{r} \Bigg(1 - \frac{1}{ikr}\Bigg).
$$

To find these fields in the near zone, the factor $kr << 1$ must remain true. If this holds then the magnetic field is

$$
    \vec{H} = \frac{ck^{2}}{4\pi} (\hat{n} \times \vec{p}) \frac{e^{i  \cdot 0}}{r} \Bigg(1 - \frac{1}{i \cdot 0}\Bigg).
$$

Then $exp(i \cdot 0) = 1$ and $1/0 = 0$, and multiply by $i/kr$ our equation becomes 

$$
    \vec{H} = \frac{ck}{4\pi} (\hat{n} \times \vec{p}) \frac{1}{r^{2}}.
$$

The same logic can be applied to the electric field. The electric field is gave again by

$$
\vec{E} = \frac{1}{4\pi \epsilon_{0}} \Bigg\{ k^{2} (\hat{n} \times \vec{p}) \times \hat{n} \frac{e^{ikr}}{r} + [3\hat{n}(\hat{n} \cdot \vec{p}) - \vec{p}] \Bigg(\frac{1}{r^{3}} - \frac{ik}{r^{2}}\Bigg )e^{ikr} \Bigg\}
$$

The first part of the equation will be zero because of the denominator. Therefore, we are left with 

\begin{align*}
\vec{E} &= \frac{1}{4\pi \epsilon_{0}} \Bigg\{[3\hat{n}(\hat{n} \cdot \vec{p}) - \vec{p}] \Bigg(\frac{i}{kr^{4}} + \frac{1}{r^{3}}\Bigg )e^{ikr} \Bigg\} \\
&= \frac{1}{4\pi \epsilon_{0}} \Bigg\{[3\hat{n}(\hat{n} \cdot \vec{p}) - \vec{p}] \Bigg(0 + \frac{1}{r^{3}}\Bigg )e^{0} \Bigg\} \\
\end{align*}

\newpage

Therefore, we are just left with

$$
\vec{E} = \frac{1}{4\pi \epsilon_{0}} [3\hat{n}(\hat{n} \cdot \vec{p}) - \vec{p}] \frac{1}{r^{3}}
$$

\clearpage

\section*{Problem 4}

The time-averaged power radiated per unit solid angle by an oscillating dipole is given by

$$
\frac{dP}{d \omega} = \frac{1}{2} Re[r^{2}\hat{n} \cdot \vec{E} \times \vec{H}^{*}]
$$

where the fields $\vec{E}$ and $\vec{H}$ in the far zone are given by

\begin{align*}
    \vec{H} &= \frac{ck^{2}}{4\pi}(\hat{n} \times \vec{p}) \frac{e^{ikr}}{r} &
    \vec{E} &= \frac{k^{2}}{4\pi \epsilon_{0}} [(\hat{n} \times \vec{p}) \times \hat{n}] \frac{e^{ikr}}{r}
\end{align*}

Substituting $\vec{H}$ and $\vec{E}$ into the equation above gives 

$$
\frac{dP}{d \omega} = \frac{1}{2} Re[r^{2}\hat{n} \cdot \frac{k^{2}}{4\pi \epsilon_{0}} \Bigg[(\hat{n} \times \vec{p}) \times \hat{n}] \frac{e^{ikr}}{r} \times \frac{ck^{2}}{4\pi}(\hat{n} \times \vec{p}) \frac{e^{ikr}}{r}\Bigg]
$$

so

\begin{align*}
    \vec{E} \times \vec{H}^{*} &= \frac{c}{\epsilon_{0}} \Bigg(\frac{k^{2}}{4\pi r}\Bigg)^{2} [\hat{n} \times (\vec{p} \times \het{n}] \times (\hat{n} \times \vec{p})^{*} \\
    &= \frac{c}{\epsilon_{0}} \hat{n} \times (\hat{n} \times \vec{p}) \cdot (\hat{n} \times \vec{p})^{*}.
\end{align*}

This can reduce to 

$$
\frac{dP}{d \omega} = \frac{1}{2} \frac{c}{\epsilon_{0}} \Bigg(\frac{k^{2}}{4\pi r}\Bigg)^{2} | (\hat{n} \times \vec{p}) \times \hat{n}|^{2}
$$

and further reduced to 

$$
\frac{dP}{d \omega} = \frac{c^{2} Z^_{0}}{32\pi^{2}} k^{4} |(\hat{n} \times \vec{p}) \times \hat{n}|^{2}
$$

where $Z_{0} = \sqrt{\mu_{0}/\epsilon_{0}}$.

\clearpage

%\section*{Appendix}

%\subsection*{}
%\lstinputlisting{.py}

\clearpage


% -------------------------- EOD -------------------------- 
\end{document}