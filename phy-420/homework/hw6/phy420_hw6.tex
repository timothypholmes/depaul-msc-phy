\documentclass[11pt]{article}

\newcommand{\HWnum}{} 
\newcommand{\StudName}{Timothy Holmes} % author
\newcommand{\CourseNum}{420}           % course number
\newcommand{\Subject}{PHY}

\usepackage{graphicx, amsmath, amssymb,fancyhdr}
\addtolength{\textwidth}{1.5in}
\addtolength{\oddsidemargin}{-2cm}
\addtolength{\evensidemargin}{-2cm}
\addtolength{\textheight}{1.6in}
\addtolength{\topmargin}{-0.7in}
\addtolength{\headsep}{-0.1in}
%\addtolength{\footskip}{-0.2in}
\pagestyle{fancy}
\cfoot{}
\lhead{\textbf{\Subject~\CourseNum~--- Homework~\HWnum}}
\rhead{\textbf{\StudName:~Page~\thepage}}

\addtolength{\parskip}{\baselineskip} % skips a line between paragraphs
\parindent 0in                        % no indent at start of paragraph

\newcommand{\dd}{\textrm{d}}
\usepackage{braket}
\usepackage{lipsum, babel}
\usepackage{blindtext}
\usepackage{graphicx}% Include figure files
\usepackage{dcolumn}% Align table columns on decimal point
\usepackage{bm}% bold math
\usepackage{listings}
\usepackage{listing}
\usepackage{supertabular}

\usepackage{setspace}\onehalfspacing
\AtBeginDocument{%
  \addtolength\abovedisplayskip{-0.5\baselineskip}%
  \addtolength\belowdisplayskip{-0.5\baselineskip}%
%  \addtolength\abovedisplayshortskip{-0.5\baselineskip}%
%  \addtolength\belowdisplayshortskip{-0.5\baselineskip}%
}


\usepackage{color} %red, green, blue, yellow, cyan, magenta, black, white
\definecolor{mygreen}{RGB}{28,172,0} % color values Red, Green, Blue
\definecolor{mylilas}{RGB}{170,55,241}



\lstset{language=Python,%
    %basicstyle=\color{red},
    breaklines=true,%
    morekeywords={matlab2tikz},
    keywordstyle=\color{blue},%
    morekeywords=[2]{1}, keywordstyle=[2]{\color{black}},
    identifierstyle=\color{black},%
    stringstyle=\color{mylilas},
    commentstyle=\color{mygreen},%
    showstringspaces=false,%without this there will be a symbol in the places where there is a space
    numbers=left,%
    numberstyle={\tiny \color{black}},% size of the numbers
    numbersep=9pt, % this defines how far the numbers are from the text
    emph=[1]{for,end,break},emphstyle=[1]\color{red}, %some words to emphasise
    %emph=[2]{word1,word2}, emphstyle=[2]{style},    
}

\begin{document}
% -------------------------- BOD -------------------------- 

\title{Homework {\HWnum}}
\author{Timothy Holmes \\ \Subject ~ \CourseNum ~ Electrodynamics II}

\maketitle

\section*{Problem 1}

The inverse Lorentz transformation equations for a frame K' traveling at velocity $v$ along the positive $x$-direction of a frame $K$ are given by

\begin{align*}
    t &= \gamma \Bigg(t' + \frac{vx'}{c^{2}} \Bigg) & x &= \gamma(x' + vt') & y &= y' & z &= z'
\end{align*}

The $x$ and $t$ equations in differential form are gave as

\begin{align*}
    dx &= \gamma(dx' - v\; dt') && \text{and} &
    dt &= \gamma\Bigg(dt' - \frac{v}{c^{2}} dx'\Bigg)
\end{align*}

where $\gamma = (1 - v^{2}/c^{2})^{-1/2}$. Now,  $u_x$ can be found by

$$
u_x = \frac{dx}{dt} = \frac{ \gamma(dx' - v\; dt')}{\gamma\big(dt' - \frac{v}{c^{2}} dx'\big)}.
$$

This can also be expressed as

$$
u_{x} = \frac{\frac{dx'}{dt'} - v}{1 - \frac{v(dx'/dt')}{c^{2}}}
$$

which can be reduced down to

$$
u_{x} = \frac{u_{x}' - v}{1 - \frac{v u_{x}'}{c^{2}}}
$$

The same approach is used to find $u_{y}$. The differential form for $y$ is $dy = dy'$. Then,  $u_{y}$ is gave as

$$
u_{y} = \frac{dy}{dt} = \frac{dy'}{\gamma\big(dt' - \frac{v}{c^{2}} dy'\big)}.
$$

This can then be written as

$$
\frac{\frac{dy'}{dt'}}{\gamma\big(1 - \frac{v(dy'/dt')}{c^{2}}\big)}
$$

and reduced to 

$$
\frac{u_{y}'}{\gamma\big(1 - \frac{v u_{y}'}{c^{2}}\big)}.
$$

\clearpage

\section*{Problem 2}

\section*{(a)}

For $u'$ parallel to $\vec{v}$, we start with 

$$
u_{\parallel} = \frac{u_{\parallel}' + v}{1+ \frac{\vec{v} \cdot \vec{u}'}{c^{2}}}
$$

The angle between $u'$ and $v$ is zero when parallel, this can also be shown by

$$
u =  \frac{u_{\parallel}' + v}{1+ \frac{|v||u|cos(0)}{c^{2}}}
$$

and reduced to

$$
u =  \frac{u' + v}{1+ \frac{v u'}{c^{2}}}
$$

\section*{(b)}

If $u' = c$, then

$$
u = \frac{u' + v}{1+ \frac{v u'}{c^{2}}} = \frac{c + v}{1+ \frac{v}{c}} = c.
$$

From the equation above we find that $u = c$, just as we have $u' =c$. Therefore, this tells us that $c$ is the maximum limit of the speed of light, this is the upper bound of speed in the universe. This limit is a key aspect and a postulate of Special Relativity. 

\section*{(c)}

Speeds $u'$ and $v$ both small compared to $c$. From part (a) we found the equation for $u$. Since $u'$ and $v$ are small compared to $c$, then the part of the equation $u'v/c^{2} = 0$ since $c$ will be much larger than $u \cdot v$. Therefore, our new equation

$$
u = \frac{u' + v}{1 + 0} = \frac{u' + v}{1} = u' + v.
$$

\clearpage

\section*{Problem 3}

The 4-velocity equation is gave by

$$
U = (\gamma_{u}c, \gamma_{u}\vec{u})
$$

where $\gamma_{u} = (1 - u^{2}/c^{2})^{-1/2}$, and $\vec{u} = d\vec{x}/dt$ is the 3-dimensional velocity.

\section*{(a)}

From the equation for $U$ above, $U$ can also be rewrote as

\begin{align*}
U &= \gamma_{u} (c - \vec{u}) && \text{or} & U^{2} &= \gamma_{u}^{2} (c^{2} - \vec{u} \cdot \vec{u})
\end{align*}

Therefore, $U^{2}$ can be wrote and reduced to

\begin{align*}
U^{2} &= \gamma_{u}^{2} (c^{2} - \vec{u} \cdot \vec{u}) \\
U^{2} &= \frac{(c^{2} - u^{2})}{(1 - u^{2}/c^{2})} \\
U^{2} &= c^{2}
\end{align*}

\section*{(b)}

We know that

$$
U = \frac{dx}{d \tau} 
$$

Therefore, the 4-acceleration 

\begin{align*}
A &= \frac{dU}{dt} \frac{dt}{d \tau} \\
A &= \gamma \frac{d}{dt} (c, u \gamma )\\
A &= \gamma \Bigg(c \frac{d \gamma}{dt}, \frac{d \gamma}{dt}u + \gamma \frac{du}{dt}\Bigg)
\end{align*}

where $du/dt = a$.

\section*{(c)}


Find the scalar product $U \cdot A$ of the 4-velocity and the 4-acceleration. After taking the derivatives in the equation for $A$, we find that

$$
U \cdot A = 0
$$

%This is true since they will not have the same components.


\clearpage

\section*{Problem 4}

The Lorentz transformation equations are given by

\begin{align*}
    x_{0}' &= \gamma(x_{0} - \beta x_{1}) \\
    x_{1}' &= \gamma(x_{1} - \beta x_{0}) \\
    x_{2}' &= x_{2} \\
    x_{3}' &= x_{3} \\
\end{align*}

where

\begin{align*}
   \beta &= \frac{v}{c} && \text{and} & \gamma = (1 - \beta^{2})^{-1/2}
\end{align*}

\section*{(a)}

If we have $\beta = tanh \zeta$, then we can use the equation for $\gamma$ to find that

$$
\gamma = (1 - \beta^{2})^{-1/2} = (1 - tanh^{2}\zeta)^{-1/2}.
$$

The alternative form for this equation is then

$$
\gamma = cosh \zeta.
$$

Now, if we know that $\gamma = cosh \zeta$, then 

$$
\gamma \beta = cosh \zeta tanh \zeta
$$

where the identity for $cosh(x)tanh(x)$ is

$$
\gamma \beta = sinh \zeta.
$$

\section*{(b)}

The Lorentz transformation equations can now be wrote as

\begin{align*}
    x_{0}' &= \gamma x_{0} - \gamma \beta x_{1} = cosh \zeta x_{0} - sinh \zeta x_{1}\\
    x_{1}' &= \gamma x_{1} - \gamma \beta x_{0} = cosh \zeta x_{1} - sinh \zeta x_{0}\\
    x_{2}' &= x_{2} \\
    x_{3}' &= x_{3} \\
\end{align*}

and in matrix for this is

\begin{align*}
\begin{pmatrix} x_{0}' \\ x_{1}' \\ x_{2}' \\ x_{3}' \end{pmatrix} =
\begin{pmatrix} 
cosh \zeta & -sinh \zeta & 0 & 0 \\ 
-sinh \zeta & cosh \zeta & 0 & 0 \\ 
0 & 0 & 1 & 0 \\ 
0 & 0 & 0 & 1 \\ 
\end{pmatrix}
\begin{pmatrix} x_{0} \\ x_{1} \\ x_{2} \\ x_{3}  \end{pmatrix} 
\end{align*}

\clearpage

%\section*{Appendix}

%\subsection*{}
%\lstinputlisting{.py}

\clearpage


% -------------------------- EOD -------------------------- 
\end{document}