\documentclass[11pt]{article}

\newcommand{\HWnum}{7} 
\newcommand{\StudName}{Timothy Holmes} % author
\newcommand{\CourseNum}{420}           % course number
\newcommand{\Subject}{PHY}

\usepackage{graphicx, amsmath, amssymb,fancyhdr}
\addtolength{\textwidth}{1.5in}
\addtolength{\oddsidemargin}{-2cm}
\addtolength{\evensidemargin}{-2cm}
\addtolength{\textheight}{1.6in}
\addtolength{\topmargin}{-0.7in}
\addtolength{\headsep}{-0.1in}
%\addtolength{\footskip}{-0.2in}
\pagestyle{fancy}
\cfoot{}
\lhead{\textbf{\Subject~\CourseNum~--- Homework~\HWnum}}
\rhead{\textbf{\StudName:~Page~\thepage}}

\addtolength{\parskip}{\baselineskip} % skips a line between paragraphs
\parindent 0in                        % no indent at start of paragraph

\newcommand{\dd}{\textrm{d}}
\usepackage{braket}
\usepackage{lipsum, babel}
\usepackage{blindtext}
\usepackage{graphicx}% Include figure files
\usepackage{dcolumn}% Align table columns on decimal point
\usepackage{bm}% bold math
\usepackage{listings}
\usepackage{listing}
\usepackage{supertabular}



\usepackage{color} %red, green, blue, yellow, cyan, magenta, black, white
\definecolor{mygreen}{RGB}{28,172,0} % color values Red, Green, Blue
\definecolor{mylilas}{RGB}{170,55,241}



\lstset{language=Python,%
    %basicstyle=\color{red},
    breaklines=true,%
    morekeywords={matlab2tikz},
    keywordstyle=\color{blue},%
    morekeywords=[2]{1}, keywordstyle=[2]{\color{black}},
    identifierstyle=\color{black},%
    stringstyle=\color{mylilas},
    commentstyle=\color{mygreen},%
    showstringspaces=false,%without this there will be a symbol in the places where there is a space
    numbers=left,%
    numberstyle={\tiny \color{black}},% size of the numbers
    numbersep=9pt, % this defines how far the numbers are from the text
    emph=[1]{for,end,break},emphstyle=[1]\color{red}, %some words to emphasise
    %emph=[2]{word1,word2}, emphstyle=[2]{style},    
}

\begin{document}
% -------------------------- BOD -------------------------- 

\title{Homework {\HWnum}}
\author{Timothy Holmes \\ \Subject ~ \CourseNum ~ Electrodynamics II}

\maketitle

\section*{Problem 1}

Starting from 
$$
A = e^{-\vec{\omega} \cdot \vec{S} - \vec{\zeta} \cdot \vec{K}}
$$
and in the case of rotation about the $x^{3}$ axis without any boost, we have $\vec{\omega} = \omega \hat{\epsilon_{3}}$ and $\vec{\zeta} = 0$. Thus, $A$ will become
$$
A = e^{-\omega S_{3}}
$$
and Taylor expanding this equation gives
\begin{align*}
    A = I + \omega S_{3} + \frac{1}{2!}(-\omega S_{3})^{2} - \frac{1}{3!}(-\omega S_{3})^{3} + ...
\end{align*}
Which can be rearranged and wrote as
$$
A = (I + S_{3}^{2}) - S_{3}sinh(\omega) - S_{3}^{2}cosh(\omega)
$$
In matrix for this is
\begin{gather*}
A = 
    \left(
    \begin{pmatrix}
    1 & 0 & 0 & 0 \\
    0 & 1 & 0 & 0 \\
    0 & 0 & 1 & 0 \\
    0 & 0 & 0 & 1 \\
    \end{pmatrix} 
    +
    \begin{pmatrix}
    0 & 0 & 0 & 0 \\
    0 & -1 & 0 & 0 \\
    0 & 0 & -1 & 0 \\
    0 & 0 & 0 & 0 \\
    \end{pmatrix} 
    \right)
    -
    \begin{pmatrix}
    0 & 0 & 0 & 0 \\
    0 & 0 & -1 & 0 \\
    0 & 1 & 0 & 0\\
    0 & 0 & 0 & 0 \\
    \end{pmatrix} 
    sinh(\omega)
    -
    \begin{pmatrix}
    0 & 0 & 0 & 0 \\
    0 & -1 & 0 & 0 \\
    0 & 0 & -1 & 0 \\
    0 & 0 & 0 & 0 \\
    \end{pmatrix} 
    cosh(\omega) \\
    =
    \begin{pmatrix}
    1 & 0 & 0 & 0 \\
    0 & 0 & 0 & 0 \\
    0 & 0 & 0 & 0 \\
    0 & 0 & 0 & 1 \\
    \end{pmatrix} 
    -
    \begin{pmatrix}
    0 & 0 & 0 & 0 \\
    0 & 0 & -sinh(\omega) & 0 \\
    0 & sinh(\omega) & 0 & 0\\
    0 & 0 & 0 & 0 \\
    \end{pmatrix} 
    -
    \begin{pmatrix}
    0 & 0 & 0 & 0 \\
    0 & -cosh(\omega) & 0 & 0 \\
    0 & 0 & -cosh(\omega) & 0 \\
    0 & 0 & 0 & 0 \\
    \end{pmatrix} \\
    =
    \begin{pmatrix}
    1 & 0 & 0 & 0 \\
    0 & cosh(\omega) & sinh(\omega) & 0 \\
    0 & -sinh(\omega) & cosh(\omega) & 0 \\
    0 & 0 & 0 & 1 \\
    \end{pmatrix} 
\end{gather*}
\clearpage

\section*{Problem 2}

Express the Lorentz. scalar $F^{\alpha \beta}F_{\alpha \beta}$ in terms of $\vec{E}$ and $\vec{B}$, where $F^{\alpha \beta}$ is given by
$$
F^{\alpha \beta} = 
\begin{pmatrix}
    0 & -E_{x} & -E_{y} & -E_{z} \\
    E_{x} & 0 & -B_{z} & B_{y} \\
    E_{y} & B_{z} & 0 & -B_{x} \\
    E_{z} & -B_{y} & B_{x} & 0
\end{pmatrix}
$$
and $F^{\alpha \beta}$ can be obtained from $F^{\alpha \beta}$ by the procedure you worked out on the class worksheet (i.e., by putting $E_{i} \rightarrow - E_{i}$, and leaving $B_{i}$ unchanged).

The tensor with two covariant indices can be found by setting all the the $E_{i}$ components to $-E_{i}$ and leaving all the $B_{i}$ components unchanged. This gives us the following
$$
F_{\alpha \beta} = 
\begin{pmatrix}
    0 & E_{x} & E_{y} & E_{z} \\
    -E_{x} & 0 & -B_{z} & B_{y} \\
    -E_{y} & B_{z} & 0 & -B_{x} \\
    -E_{z} & -B_{y} & B_{x} & 0
\end{pmatrix}
$$
Therefore, 
\begin{align*}
F^{\alpha \beta} F_{\alpha \beta} &= 
 \begin{pmatrix}
    0 & -E_{x} & -E_{y} & -E_{z} \\
    E_{x} & 0 & -B_{z} & B_{y} \\
    E_{y} & B_{z} & 0 & -B_{x} \\
    E_{z} & -B_{y} & B_{x} & 0
\end{pmatrix}
\begin{pmatrix}
    0 & E_{x} & E_{y} & E_{z} \\
    -E_{x} & 0 & -B_{z} & B_{y} \\
    -E_{y} & B_{z} & 0 & -B_{x} \\
    -E_{z} & -B_{y} & B_{x} & 0
\end{pmatrix} \\
&=\left(\begin{array}{cccc}
E_{x}^{2}+E_{y}^{2}+E_{z}^{2} & -E_{y} B_{z}+E_{z} B_{y} & E_{x} B_{z}-E_{z} B_{x} & -E_{x} B_{y}+E_{y} B_{x} \\
B_{z} E_{y}-B_{y} E_{z} & E_{x}^{2}-B_{z}^{2}-B_{y}^{2} & E_{x} E_{y}+B_{y} B_{x} & E_{x} E_{z}+B_{z} B_{x} \\
-B_{z} E_{x}+B_{x} E_{z} & E_{y} E_{x}+B_{x} B_{y} & E_{y}^{2}-B_{z}^{2}-B_{x}^{2} & E_{y} E_{z}+B_{z} B_{y} \\
B_{y} E_{x}-B_{x} E_{y} & E_{z} E_{x}+B_{x} B_{z} & E_{z} E_{y}+B_{y} B_{z} & E_{z}^{2}-B_{y}^{2}-B_{x}^{2}
\end{array}\right)
\end{align*}

\clearpage

\section*{Problem 3}
Consider the fundamental matrices $S_{1}, S_{2}, S_{3}, K_{1}, K_{2}, K_{3}$ written in equation (11.19) in Jackson. By explicit matrix multiplication, find the commutators 
\begin{align*}
    & [S_{2}, S_{3}], & [S_{2}, K_{3}], && \text{and} && [K_{2}, K_{3}]
\end{align*}

\begin{gather*}
    [S_{2}, S_{3}] = S_{2}S_{3} - S_{3}S_{2} \\
    =
    \begin{pmatrix}
    0 & 0 & 0 & 0 \\
    0 & 0 & 0 & 1 \\
    0 & 0 & 0 & 0\\
    0 & -1 & 0 & 0 \\
    \end{pmatrix} 
    \begin{pmatrix}
    0 & 0 & 0 & 0 \\
    0 & 0 & -1 & 0 \\
    0 & 1 & 0 & 0\\
    0 & 0 & 0 & 0 \\
    \end{pmatrix} 
    -
    \begin{pmatrix}
    0 & 0 & 0 & 0 \\
    0 & 0 & -1 & 0 \\
    0 & 1 & 0 & 0\\
    0 & 0 & 0 & 0 \\
    \end{pmatrix} 
    \begin{pmatrix}
    0 & 0 & 0 & 0 \\
    0 & 0 & 0 & 1 \\
    0 & 0 & 0 & 0\\
    0 & -1 & 0 & 0 \\
    \end{pmatrix} \\
    =
    \begin{pmatrix}
    0 & 0 & 0 & 0 \\
    0 & 0 & 0 & 0 \\
    0 & 0 & 0 & 0\\
    0 & 0 & 1 & 0 \\
    \end{pmatrix}
    -
    \begin{pmatrix}
    0 & 0 & 0 & 0 \\
    0 & 0 & 0 & 0 \\
    0 & 0 & 0 & 1\\
    0 & 0 & 0 & 0 \\
    \end{pmatrix} \\
    =
    \begin{pmatrix}
    0 & 0 & 0 & 0 \\
    0 & 0 & 0 & 0 \\
    0 & 0 & 0 & -1\\
    0 & 0 & 1 & 0 \\
    \end{pmatrix} = S_{1}
\end{gather*}

\begin{gather*}
    [S_{2}, K_{3}] = S_{2}K_{3} - K_{3}S_{2} \\
    =
    \begin{pmatrix}
    0 & 0 & 0 & 0 \\
    0 & 0 & 0 & 1 \\
    0 & 0 & 0 & 0\\
    0 & -1 & 0 & 0 \\
    \end{pmatrix} 
\left(\begin{array}{llll}
0 & 0 & 0 & 1 \\
0 & 0 & 0 & 0 \\
0 & 0 & 0 & 0 \\
1 & 0 & 0 & 0
\end{array}\right)
    -
\left(\begin{array}{llll}
0 & 0 & 0 & 1 \\
0 & 0 & 0 & 0 \\
0 & 0 & 0 & 0 \\
1 & 0 & 0 & 0
\end{array}\right)
    \begin{pmatrix}
    0 & 0 & 0 & 0 \\
    0 & 0 & 0 & 1 \\
    0 & 0 & 0 & 0\\
    0 & -1 & 0 & 0 \\
    \end{pmatrix} \\
    =
    \begin{pmatrix}
    0 & 0 & 0 & 0 \\
    1 & 0 & 0 & 0 \\
    0 & 0 & 0 & 0\\
    0 & 0 & 0 & 0 \\
    \end{pmatrix}
    -
    \begin{pmatrix}
    0 & -1 & 0 & 0 \\
    0 & 0 & 0 & 0 \\
    0 & 0 & 0 & 0\\
    0 & 0 & 0 & 0 \\
    \end{pmatrix} \\
    =
    \begin{pmatrix}
    0 & 1 & 0 & 0 \\
    1 & 0 & 0 & 0 \\
    0 & 0 & 0 & 0 \\
    0 & 0 & 0 & 0 \\
    \end{pmatrix} = K_{1}
\end{gather*}

\begin{gather*}
    [K_{2}, K_{3}] = K_{2}K_{3} - K_{3}K_{2} \\
    =
\left(\begin{array}{llll}
0 & 0 & 1 & 0 \\
0 & 0 & 0 & 0 \\
1 & 0 & 0 & 0 \\
0 & 0 & 0 & 0
\end{array}\right)
\left(\begin{array}{llll}
0 & 0 & 0 & 1 \\
0 & 0 & 0 & 0 \\
0 & 0 & 0 & 0 \\
1 & 0 & 0 & 0
\end{array}\right)
    -
\left(\begin{array}{llll}
0 & 0 & 0 & 1 \\
0 & 0 & 0 & 0 \\
0 & 0 & 0 & 0 \\
1 & 0 & 0 & 0
\end{array}\right)
\left(\begin{array}{llll}
0 & 0 & 1 & 0 \\
0 & 0 & 0 & 0 \\
1 & 0 & 0 & 0 \\
0 & 0 & 0 & 0
\end{array}\right) \\
    =
    \begin{pmatrix}
    0 & 0 & 0 & 0 \\
    0 & 0 & 0 & 0 \\
    0 & 0 & 0 & 1 \\
    0 & 0 & 0 & 0 \\
    \end{pmatrix}
    -
    \begin{pmatrix}
    0 & 0 & 0 & 0 \\
    0 & 0 & 0 & 0 \\
    0 & 0 & 0 & 0 \\
    0 & 0 & 1 & 0 \\
    \end{pmatrix} \\
    =
    \begin{pmatrix}
    0 & 0 & 0 & 0 \\
    0 & 0 & 0 & 0 \\
    0 & 0 & 0 & 1 \\
    0 & 0 & -1 & 0 \\
    \end{pmatrix} = -S_{1}
\end{gather*}

\clearpage


\section*{Problem 4}

\subsection*{(a)}

Recall that the fields $\vec{E}$ and $\vec{B}$ can be expressed in terms of the potentials as 
\begin{align*}
    \vec{E} = - \frac{1}{c} \frac{\partial \vec{A}}{\partial t} - \vec{\nabla} \Phi && \text{and} && \vec{B} &= \vec{\nabla} \times \vec{A}
\end{align*}
The components of $\vec{E}$ and $\vec{B}$ using the $\partial^{\alpha}$ notation for the $x$ component is
\begin{align*}
    \vec{E_{x}} &= -\frac{1}{c}\frac{\partial A_{x}}{\partial t} - \frac{\partial \Phi}{\partial x} = -(\partial^{0}A^{1} - \partial^{1}A^{0}) 
    && \text{and} &&
    \vec{B_{x}} &= \frac{\partial A_{z}}{\partial y} - \frac{\partial A^{y}}{\partial z} = -(\partial^{2}A^{3} - \partial^{3}A^{2})
\end{align*}
where
$$
\partial^{\alpha} = \Bigg(\frac{\partial}{\partial x^{0}}, -\vec{\nabla}\Bigg).
$$
For the $y$ component
\begin{align*}
    \vec{E_{y}} &= -\frac{1}{c}\frac{\partial A_{y}}{\partial t} - \frac{\partial \Phi}{\partial y} = -(\partial^{0}A^{2} - \partial^{2}A^{0}) 
    && \text{and} &&
    \vec{B_{y}} &= \frac{\partial A_{z}}{\partial y} - \frac{\partial A^{y}}{\partial z} = -(\partial^{1}A^{3} - \partial^{3}A^{1})
\end{align*}
where
$$
\partial^{\alpha} = \Bigg(\frac{\partial}{\partial y^{0}}, -\vec{\nabla}\Bigg).
$$
For the $z$ component 
\begin{align*}
    \vec{E_{z}} &= -\frac{1}{c}\frac{\partial A_{z}}{\partial t} - \frac{\partial \Phi}{\partial z} = -(\partial^{0}A^{3} - \partial^{3}A^{0}) 
    && \text{and} &&
    \vec{B_{z}} &= \frac{\partial A_{z}}{\partial y} - \frac{\partial A^{y}}{\partial z} = -(\partial^{1}A^{2} - \partial^{2}A^{1})
\end{align*}
where
$$
\partial^{\alpha} = \Bigg(\frac{\partial}{\partial z^{0}}, -\vec{\nabla}\Bigg).
$$

\subsection*{(b)}
The element obtained above are the elements of the field tensor
$$
F^{\alpha \beta} = \partial^{\alpha}A^{\beta} - \partial^{\beta}A^{\alpha}
$$
The following matrix can be used to match terms:
\begin{align*}
F^{\alpha \beta} = 
\left(\begin{array}{llll}
F^{00} & F^{01} & F^{02} & F^{03} \\
F^{10} & F^{11} & F^{12} & F^{13} \\
F^{20} & F^{21} & F^{22} & F^{23} \\
F^{30} & F^{31} & F^{32} & F^{33}
\end{array}\right)
= 
\begin{pmatrix}
    0 & -E_{x} & -E_{y} & -E_{z} \\
    E_{x} & 0 & -B_{z} & B_{y} \\
    E_{y} & B_{z} & 0 & -B_{x} \\
    E_{z} & -B_{y} & B_{x} & 0
\end{pmatrix}
\end{align*}
Thus, we have $F^{\alpha 0}$ which gives
\begin{align*}
F^{00} &= (\partial^{0} A^{0} - \partial^{0}A^{0}) = 0 &
F^{10} &= (\partial^{1} A^{0} - \partial^{0}A^{1}) = E_{x} \\
F^{20} &= (\partial^{2} A^{0} - \partial^{0}A^{2}) = E_{y} &
F^{30} &= (\partial^{3} A^{0} - \partial^{0}A^{3}) = E_{z} \\
\end{align*}
$F^{\alpha 1}$ which gives
\begin{align*}
F^{01} &= (\partial^{0} A^{1} - \partial^{1}A^{0}) = -E_{x} &
F^{11} &= (\partial^{1} A^{1} - \partial^{1}A^{1}) = 0 \\
F^{21} &= (\partial^{2} A^{1} - \partial^{1}A^{2}) = B_{z} &
F^{31} &= (\partial^{3} A^{1} - \partial^{1}A^{3}) = -B_{y} \\
\end{align*}
$F^{\alpha 2}$ which gives
\begin{align*}
F^{02} &= (\partial^{0} A^{2} - \partial^{2}A^{0}) = -E_{y} &
F^{12} &= (\partial^{1} A^{2} - \partial^{2}A^{1}) = -B_{z} \\
F^{22} &= (\partial^{2} A^{2} - \partial^{2}A^{2}) = 0 &
F^{32} &= (\partial^{3} A^{2} - \partial^{2}A^{3}) = B_{x} \\
\end{align*}
$F^{\alpha 3}$ which gives
\begin{align*}
F^{03} &= (\partial^{0} A^{3} - \partial^{3}A^{0}) = -E_{z} &
F^{13} &= (\partial^{1} A^{3} - \partial^{3}A^{1}) = -B_{y} \\
F^{23} &= (\partial^{2} A^{3} - \partial^{3}A^{2}) = -B_{x} &
F^{33} &= (\partial^{3} A^{3} - \partial^{3}A^{3}) = 0 \\
\end{align*}

\clearpage


%\section*{Appendix}

\subsection*{}
%\lstinputlisting{.py}

\clearpage


% -------------------------- EOD -------------------------- 
\end{document}