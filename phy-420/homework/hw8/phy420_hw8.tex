\documentclass[11pt]{article}

\newcommand{\HWnum}{8} 
\newcommand{\StudName}{Timothy Holmes} % author
\newcommand{\CourseNum}{420}           % course number
\newcommand{\Subject}{PHY}

\usepackage{graphicx, amsmath, amssymb,fancyhdr}
\addtolength{\textwidth}{1.5in}
\addtolength{\oddsidemargin}{-2cm}
\addtolength{\evensidemargin}{-2cm}
\addtolength{\textheight}{1.6in}
\addtolength{\topmargin}{-0.7in}
\addtolength{\headsep}{-0.1in}
%\addtolength{\footskip}{-0.2in}
\pagestyle{fancy}
\cfoot{}
\lhead{\textbf{\Subject~\CourseNum~--- Homework~\HWnum}}
\rhead{\textbf{\StudName:~Page~\thepage}}

\addtolength{\parskip}{\baselineskip} % skips a line between paragraphs
\parindent 0in                        % no indent at start of paragraph

\newcommand{\dd}{\textrm{d}}
\usepackage{braket}
\usepackage{lipsum, babel}
\usepackage{blindtext}
\usepackage{graphicx}% Include figure files
\usepackage{dcolumn}% Align table columns on decimal point
\usepackage{bm}% bold math
\usepackage{listings}
\usepackage{listing}
\usepackage{supertabular}



\usepackage{color} %red, green, blue, yellow, cyan, magenta, black, white
\definecolor{mygreen}{RGB}{28,172,0} % color values Red, Green, Blue
\definecolor{mylilas}{RGB}{170,55,241}



\lstset{language=Python,%
    %basicstyle=\color{red},
    breaklines=true,%
    morekeywords={matlab2tikz},
    keywordstyle=\color{blue},%
    morekeywords=[2]{1}, keywordstyle=[2]{\color{black}},
    identifierstyle=\color{black},%
    stringstyle=\color{mylilas},
    commentstyle=\color{mygreen},%
    showstringspaces=false,%without this there will be a symbol in the places where there is a space
    numbers=left,%
    numberstyle={\tiny \color{black}},% size of the numbers
    numbersep=9pt, % this defines how far the numbers are from the text
    emph=[1]{for,end,break},emphstyle=[1]\color{red}, %some words to emphasise
    %emph=[2]{word1,word2}, emphstyle=[2]{style},    
}

\begin{document}
% -------------------------- BOD -------------------------- 

\title{Homework {\HWnum}}
\author{Timothy Holmes \\ \Subject ~ \CourseNum ~ Electrodynamics II}

\maketitle

\section*{Problem 1}

To prove that a pure magnetic field in one frame cannot be a pur electric field in another frame we can do the following:
For a general Lorentz transformation from $K$ to a frame $K'$ moving with velocity $\vec{v}$ relative to $K$, the transformation of the fields is
\begin{align*}
    \vec{E}' &= \gamma(\vec{E} + \vec{\beta} \times \vec{B}) - \frac{\gamma^{2}}{\gamma + 1} \vec{\beta} (\vec{\beta} \cdot \vec{E}) \\
    \vec{B}' &= \gamma(\vec{B} - \vec{\beta} \times \vec{E}) - \frac{\gamma^{2}}{\gamma + 1} \vec{\beta} (\vec{\beta} \cdot \vec{B}) 
\end{align*}
From here we will $\vec{E} = 0$ so the transformation is 
$$
\vec{E}' &= \gamma(0 + \vec{\beta} \times \vec{B}) - \frac{\gamma^{2}}{\gamma + 1} \vec{\beta} (\vec{\beta} \cdot 0) = \gamma(\vec{\beta} \times \vec{B}).
$$
Then for the magnetic field, when  $\vec{E} = 0$, we have
$$
\vec{B}' &= \gamma(\vec{B} - \vec{\beta} \times 0) - \frac{\gamma^{2}}{\gamma + 1} \vec{\beta} (\vec{\beta} \cdot \vec{B}) = \gamma \vec{B}  - \frac{\gamma^{2}}{\gamma + 1} \vec{\beta} (\vec{\beta} \cdot \vec{B}).
$$
Now, taking the cross product of both side with $\vec{\beta}$ will give
$$
\vec{\beta} \times \vec{B}' = \gamma(\vec{\beta} \times \vec{B})  - \frac{\gamma^{2}}{\gamma + 1} \beta \times \vec{\beta} (\vec{\beta} \cdot \vec{B}).
$$
Now, $\beta \times \beta$ is zero, so the final result is
$$
\vec{\beta} \times \vec{B}' = \gamma(\vec{\beta} \times \vec{B}) = \vec{E}'.
$$
If we substitute this into the equation we have for the magnetic field we get
$$
\vec{E}' &= \gamma(\vec{\beta} \times \vec{B}).
$$
Therefore, we get
$$
\vec{E}' = \vec{\beta} \times \vec{B}'
$$
Now if we set $\vec{E}' = 0$ we have
$$
0 = \vec{\beta} \times \vec{B}' = 0.
$$
This satisfies a purely magnetic field in frame $K$ to be a purely electric field in field $K'$.
\clearpage

\section*{Problem 2}

We can show that the electric field in terms of the present position of the charge is given by
$$
\vec{E} = \frac{q\vec{r}}{r^{3}\gamma^{2}(1 - \beta^{2} sin^{2} \phi)^{3/2}}
$$
where $r$ is the radial distance from the present position of the charge to the observer, and the angle $\phi = cos^{-1}(\hat{n} \cdot \hat{v})$ is between the direction of $\hat{n}$ and $\hat{v}$, where $\hat{n}$ is a unit radial vector from the present position of the charge to the observation point, and $\vec{v}$ is along the positive $x_{1}$-axis.

The point $P$ in frame $K'$ has coordinates
\begin{align*}
    x_{1} &= -vt' &  x_{2} &= b &  x_{3} &= 0.
\end{align*}
Then from Coulomb's law we get
\begin{align*}
    E'_{1} &= -\frac{qvt'}{r'^{3}} &  E'_{2} &= \frac{qb}{r'^{3}} &  E'_{3} &= 0 \\
    B'_{1} &= 0 &  B'_{2} &= 0 &  B'_{3} &= 0.
\end{align*}
To write this in the $K$ frame we have $r'^{2} = b^{2} + v^{2}t'^{2}$ and $ct' = \gamma c t$. Therefore, $r'^{2} = b^{2} + v^{2} \gamma^{2} t^{2}$. and we now have that
\begin{align*}
    E'_{1} &= -\frac{q\gamma vt'}{(b^{2} + v^{2} \gamma^{2} t^{2})^{3/2}} &  E'_{2} &= -\frac{q}{(b^{2} + v^{2} \gamma^{2} t^{2})^{3/2}} &  E'_{3} &= 0.
\end{align*}
Inverting these fields gives us
\begin{align*}
    E_{1} &= -\frac{q\gamma vt'}{(b^{2} + v^{2} \gamma^{2} t^{2})^{3/2}} &  E_{2} &= -\frac{q\gamma b}{(b^{2} + v^{2} \gamma^{2} t^{2})^{3/2}} &  E_{3} &= 0.
\end{align*}
Using Biot-Savart's law we find that $vb = v r sin \psi$. The electric field is directed along $\hat{n}$, this can be found by 
$$
\frac{E_{1}}{E_{2}} = -\frac{vt}{b}.
$$
We can solve the distance between point $P$ and charge $q$. We find that the distance between these points is $r^{2} - \beta^{2} r^{2} sin^{2} \psi$ where $b^{2} = \beta^{2} r^{2} sin^{2} \psi$ and $r^{2} = b^{2} + v^{2} t^{2}$. Or equivalently this distance is $\gamma^{-2}(b^{2} + \gamma^{2} v^{2} t^{2})^{1/2}$ which happens to be the denominator to or electric field. Furthermore, the magnitude of our field is
\begin{align*}
E &= \frac{q\gamma}{(b^{2} + v^{2} \gamma^{2} t^{2})^{3/2}}(b^{2} + v^{2} t^{2})^{1/2} \\
&= \frac{q \vec{r}}{\gamma^{2}(r^{2} - \beta^{2} r^{2} sin^{2} \psi)^{3/2}}
\end{align*}
rearranging some terms we eventually arrive at
$$
\vec{E} = \frac{q\vec{r}}{r^{3}\gamma^{2}(1 - \beta^{2} sin^{2} \psi)^{3/2}}.
$$

\clearpage

%\section*{Appendix}

\subsection*{}
%\lstinputlisting{.py}

\clearpage


% -------------------------- EOD -------------------------- 
\end{document}