\documentclass[11pt]{article}

\newcommand{\HWnum}{6} 
\newcommand{\StudName}{Timothy Holmes} % author
\newcommand{\CourseNum}{411}           % course number
\newcommand{\Subject}{PHY}

\usepackage{graphicx, amsmath, amssymb,fancyhdr}
\addtolength{\textwidth}{1.5in}
\addtolength{\oddsidemargin}{-2cm}
\addtolength{\evensidemargin}{-2cm}
\addtolength{\textheight}{1.6in}
\addtolength{\topmargin}{-0.7in}
\addtolength{\headsep}{-0.1in}
%\addtolength{\footskip}{-0.2in}
\pagestyle{fancy}
\cfoot{}
\lhead{\textbf{\Subject~\CourseNum~--- Homework~\HWnum}}
\rhead{\textbf{\StudName:~Page~\thepage}}

\addtolength{\parskip}{\baselineskip} % skips a line between paragraphs
\parindent 0in                        % no indent at start of paragraph

\newcommand{\dd}{\textrm{d}}
\usepackage{braket}
\usepackage{lipsum, babel}
\usepackage{blindtext}
\usepackage{tikz}

\begin{document}
% -------------------------- BOD -------------------------- 

\title{Homework {\HWnum}}
\author{Timothy Holmes \\ \Subject~411 Electrodynamics I}

\maketitle

\section*{Problem 1}

The time-averaged potential of a neutral hydrogen atom is given by

$$
\Phi = \frac{q}{4\pi \epsilon_{0}} \frac{e^{-\alpha r}}{r}\Bigg(1 + \frac{\alpha r}{2}\Bigg)
$$

Starting with the Poisson equation in spherical coordinates

$$
\nabla^{2} \Phi = -\frac{\rho}{\epsilon_{0}}
$$

entering the spherical coordinates (gave in the back of Jackson) gives

$$
\frac{1}{r^{2}}\frac{\partial}{\partial r} \Bigg(r^{2} \frac{\partial \Phi}{\partial r} \Bigg) + \frac{1}{r^{2}sin\theta} \frac{\partial}{\partial \theta} \Bigg(sin\theta \frac{\partial \Phi}{\partial \theta} \Bigg) + \frac{1}{r^{2}sin^{2}\theta} \frac{\partial^{2}\Phi}{\partial \phi^{2}} = -\frac{\rho}{\epsilon_{0}}.
$$

Looking back at the potential $\Phi$ it is clear that it only depends of $r$. There are no appearances of $\theta$ or $\phi$. Therefore, the spherical component of this. equation can be reduced down to only have terms with $r$ where the derivatives won't be zero, since the derivatives of $\theta$ and $\phi$ will just be zero. Therefore, the equation will become

$$
\frac{1}{r^{2}}\frac{\partial}{\partial r}  \Bigg(r^{2} \frac{\partial \Phi}{\partial r} \Bigg) + 0 + 0 = -\frac{\rho}{\epsilon_{0}}.
$$

Now, entering the potential yields 

$$
\frac{1}{r^{2}}\frac{\partial}{\partial r}  \Bigg(r^{2} \frac{\partial}{\partial r} \Bigg( \frac{q}{4\pi \epsilon_{0}} \frac{e^{-\alpha r}}{r}\Bigg(1 + \frac{\alpha r}{2}\Bigg) \Bigg)\Bigg) = -\frac{\rho}{\epsilon_{0}}.
$$

To be able to differentiate the equation can be setup as

$$
\frac{1}{r^{2}}\frac{\partial}{\partial r}  \Bigg(r^{2} \Bigg[ \frac{q}{4\pi \epsilon_{0}} \Bigg(\frac{\partial}{\partial r} \frac{e^{-\alpha r}}{r} + \frac{\partial}{\partial r} \frac{\alpha e^{-\alpha r} }{2}\Bigg) \Bigg]\Bigg) = -\frac{\rho}{\epsilon_{0}}.
$$

The derivatives from the equation above are

\begin{align*}
    \frac{\partial}{\partial r} \frac{e^{-\alpha r}}{r} &= -\frac{e^{-\alpha r}(\alpha r + 1)}{r^{2}} & 
    \frac{\partial}{\partial r} \frac{\alpha e^{-\alpha r} }{2} &= -\frac{\alpha^{2} e^{-\alpha r}}{2}
\end{align*}

Therefore, the equation becomes

\begin{align*}
    -\frac{\rho}{\epsilon_{0}} &= \frac{q}{4\pi\epsilon_{0}}\frac{1}{r^{2}}\frac{\partial}{\partial r}\Bigg(-\alpha r e^{\alpha r} + r^{2}e^{-\alpha r} 
    \frac{\partial}{\partial r} \frac{1}{r} - \frac{\alpha^{2}r^{2}e^{-\alpha r}}{r} \Bigg) \\
    -\rho &= \frac{q}{4\pi}\frac{1}{r^{2}}\frac{\partial}{\partial r}\Bigg(-\alpha r e^{\alpha r} + r^{2}e^{-\alpha r} 
    \frac{\partial}{\partial r} \frac{1}{r} - \frac{\alpha^{2}r^{2}e^{-\alpha r}}{r} \Bigg)
\end{align*}

Taking the final derivatives and sorting the equation gives

$$
\rho = -\frac{q\alpha^{3}e^{-\alpha r} }{8\pi}- \frac{q e^{-\alpha r} }{4\pir^{2}} \frac{\partial}{\partial r} \Bigg(r^{2} \frac{\pratial}{\partial r} \frac{1}{r}\Bigg).
$$

Now, if we have a value of $r$ that is very small or is zero, this equation will not work. part of the equation will just be zero and go away. However, this can not happen. A fix around this is to use the relation that Jackson has on page 35 just before equation 1.31. The relation shows that 

$$
\nabla^{2}\Bigg(\frac{1}{r}\Bigg) = -4\pi\delta(x).
$$

Using this relationship, all of 

$$
\frac{\partial}{\partial r} \Bigg(r^{2} \frac{\pratial}{\partial r} \frac{1}{r}\Bigg) = -4\pi\delta(x).
$$

Thus, the piece-wise function we have is 

\[ \rho = \begin{cases} 
      -\frac{q\alpha^{3} }{8\pi} + q\delta(r)  & for \; r \approx 0 \\
      -\frac{q\alpha^{3}e^{-\alpha r}}{8\pi} & for \; r > 0 \\
      -\frac{q\alpha^{3}e^{-\alpha r}}{8\pi} + q\delta(r) & for \; all \; r \\
   \end{cases}
\]



%\begin{align*}
%-\frac{\rho}{\epsilon_{0}} &= \frac{1}{r^{2}}\frac{\partial}{\partial r} 
%\Bigg(r^{2} 
%\Bigg[ \frac{q}{4\pi \epsilon_{0}} \Bigg(-\frac{e^{-\alpha r}(\alpha r + 1)}{r^{2}} -\frac{\alpha^{2} e^{-\alpha r}}{2}\Bigg) \Bigg]\Bigg) \\
%\rho &= 
% \frac{1}{r^{2}} \frac{\partial}{\partial r} \frac{q  e^{-\alpha r}(\alpha r + 1)}{4\pi} + 
%\frac{1}{r^{2}} \frac{\partial}{\partial r}
%\frac{q r^{2} \alpha^{2} e^{-\alpha r}}{8 \pi}
%\end{align*}


\clearpage

\section*{Problem 2}

\iffalse

\begin{center}
\begin{tikzpicture}
    \fill[gray]
        (0, 0) rectangle (2, 12)
        (8, 0) rectangle (10, 12)
        (0, 0) rectangle (10, 2);

    \node[anchor = north] at (2, 2)   {\scriptsize 0};
    \node[anchor = north] at (8, 2)   {\scriptsize $a$};
    \node[anchor = north] at (10.2, 2)   {\scriptsize $x$};
    \node[anchor = north] at (2, 12.5)   {\scriptsize $y$};
    \node[anchor = south] at (5, 2.5) {\scriptsize $\Phi = V$}
    \node[anchor = south] at (2.5, 6) {\scriptsize $\Phi = 0$};
    \node[anchor = south] at (7.5, 6) {\scriptsize $\Phi = 0$};
    \node[anchor = south] at (5, 11) {\scriptsize $\Phi = 0$};
    \draw[->] (5, 11.5) to (5, 12);

    \draw[->] (2, 2) to (10, 2);
    \draw[->]  (2, 2) to (2, 12);
    \draw[->]  (8, 2) to (8, 12);
\end{tikzpicture}
\end{center}
\fi

The two-dimensional Laplace equation is gave by

$$
\frac{\partial \Phi}{\partial x^{2}} + \frac{\partial^{2}\Phi}{\partial y^{2}} = 0
$$

while the boundary conditions are $\Phi = 0$ at $x = 0$ and $x = a$, $\Phi = V$ at $y = 0$ for $0 \leq x \leq a$, and $\Phi \rightarrow 0$ for large $y$.

To begin separating the variables and to apply the boundary conditions, this begins with the form

$$
\Phi(x, y) = X(x)Y(y)
$$

which can be entered back into the Laplace equation as

$$
\frac{\partial }{\partial x^{2}} (X(x)Y(y))+ \frac{\partial^{2}}{\partial y^{2}} (X(x)Y(y))= 0.
$$

From here, the variables are separated to find a familiar equation

$$
\frac{1}{X(x)}\frac{\partial^{2} X(x)}{\partial x^{2}} + \frac{1}{Y(y)}\frac{\partial^{2} Y(y)}{\partial y^{2}} = 0
$$

of which the variables are now independent of one another. Now, part of the equation can be set as a constant such that

$$
\frac{1}{X(x)}\frac{\partial^{2} X(x)}{\partial x^{2}} = -Z
$$

which implies that

$$
\frac{1}{Y(y)}\frac{\partial^{2} Y(y)}{\partial y^{2}} = Z.
$$

Setting the new coefficient $Z$ to $Z = k^{2}$ for the general solutions. The general solution for $X(x)$ is

$$
X(x) = Asin(kx) + Bcos(kx)
$$

while the general solution for $Y(y)$ is

$$
Y(y) = Ce^{ky} + De^{-ky}
$$

where $k = n\pi/a$. With the general solutions now set, boundary conditions can be applied. The following boundary conditions are satisfied. 

\begin{align*}
    \Phi(x = 0, y) &= 0 \rightarrow Asin(0) + Bcos(0) \rightarrow B = 0\\  
    \Phi(x = a, y) &= 0 \rightarrow Asin(ka) \\
    \Phi(x, y = 0) &= V \rightarrow Ce^{k0} + De^{-k0 }  \rightarrow C + D\\
    \Phi(x, y = \infty) &= 0 \rightarrow C \rightarrow \infty \; \text{and} \; D \rightarrow 0\\
\end{align*}

Therefore, the surviving terms are

$$
Asin(ka) \; \text{and} \; De^{-n\pi y/a}.
$$


Thus, the solution to the Laplace's equation is 

$$
\Phi(x, y) = \sum_{n = 1}^{\infty} A_{n}sin\Big(\frac{n\pi x}{a}\Big) e^{-n\pi y/a}.
$$

\clearpage

\section*{Problem 3}

Show that 

\[ A_{n} = \begin{cases} 
       \frac{4V}{\pi n} & for \; n \; odd \\
      0 & for \; n \; even. \\
   \end{cases}
\]

Starting with the equation in Problem 2

$$
\Phi(x, y) = \sum_{n = 1}^{\infty} A_{n}sin\Big(\frac{n\pi x}{a}\Big) e^{-n\pi y/a}
$$

Using the boundary condition where $y = 0$ the equation can be set up such that

$$
\Phi(x, y = 0) =  \sum_{n = 1}^{\infty} A_{n}sin\Big(\frac{n\pi x}{a}\Big) e^{-n\pi 0/a} = V
$$

which reduces down to 

$$
\Phi(x, y = 0) =  \sum_{n = 1}^{\infty} A_{n}sin\Big(\frac{n\pi x}{a}\Big) = V.
$$

Multiplying both sides by $sin(nx)$ and integrating between $0 \leq x \leq a$ creates

\begin{align*}
\sum_{n = 1}^{\infty} A_{n} \int_{0}^{a} sin(nx) sin\Big(\frac{n\pi x}{a}\Big) dx &=  \int_{0}^{a} sin(nx) V dx. \\
A_{n} \frac{a}{2} \delta &= \int_{0}^{a} sin(nx) V dx
\end{align*}

Integrating the LHS gives us

$$
A_{n} = \frac{2}{a} \int_{0}^{a} sin\Big(\frac{n\pi x}{a}\Big) V dx.
$$

Integrating the equation above gives 

\begin{align*}
    A_{n} &= \frac{2}{a} \int_{0}^{a} sin\Big(\frac{n\pi x}{a}\Big) V dx \\
          &= \Bigg[-\frac{2V}{\pi n} (cos\Big(\frac{n\pi x}{a}\Big) - 1) \Bigg]_{0}^{a} \\
          &= -\frac{2V}{\pi n} (cos(\pi n) - 1).
\end{align*}

When $n$ is even we have

$$
A_{n} = -\frac{2V}{\pi n} (1 - 1) = 0
$$

and when $n$ is odd we have

$$
A_{n} = -\frac{2V}{\pi n} (-1 - 1) = \frac{4V}{\pi n}.
$$

Thus,

\[ A_{n} = \begin{cases} 
       \frac{4V}{\pi n} & for \; n \; odd \\
      0 & for \; n \; even. \\
   \end{cases}
\]

or

$$
\Phi(x, y) = \sum_{n = odd}^{\infty} \frac{4V}{\pi n}sin\Big(\frac{n\pi x}{a}\Big) e^{-n\pi y/a}
$$

\clearpage

\section*{Problem 4}

Rodrigues' formula is gave by:

$$
P_{l}(x) = \frac{1}{2^{l}l!}\frac{d^{l}}{dx^{l}}(x^{2} - 1)^{l}.
$$

First, we can rewrite the expression as $P_{l+1}$, this just adds a 1 wherever a $l$ is, which gives

$$
P_{l+1}(x) = \frac{1}{2^{l+1}(l+1)!}\Big(\frac{d}{dx}\Big)^{l+1}(x^{2} - 1)^{l+1}.
$$

Which can be also wrote as

$$
P_{l+1}(x) = \frac{1}{2(l+1)}\frac{1}{2^{l}l!}\Big(\frac{d}{dx}\Big)^{l}\Big(\frac{d^{2}}{dx^{2}}(x^{2} - 1)^{l+1}\Big).
$$

differentiating the term in the parenthesis give

\begin{align*}
\frac{d^{2}}{dx^{2}}(x^{2} - 1)^{l+1} &= 4l(l+1)x^{2}(x^{2} - 1)^{l-1}+2(l+1)(x^{2} - 1)^{l}. \\
\end{align*}

This can now be entered back into the working equation

$$
P_{l+1}(x) = \frac{1}{2(l+1)}\frac{1}{2^{l}l!}\Big(\frac{d}{dx}\Big)^{l}\Big(4l(l+1)x^{2}(x^{2} - 1)^{l-1}+2(l+1)(x^{2} - 1)^{l}\Big).
$$

Factoring in the $1/(2(l+1))$ will reduce the equation to 

\begin{align*}
P_{l+1}(x) &= \frac{1}{2^{l}l!}\Big(\frac{d}{dx}\Big)^{l}\Big(2lx^{2}(x^{2} - 1)^{l-1}+(x^{2} - 1)^{l}\Big) \\
 &= \frac{1}{2^{l}l!}\Big(\frac{d}{dx}\Big)^{l}\Big(2lx^{2}(x^{2} - 1)^{l-1}+(x^{2} - 1)^{l}\Big) \\
 &= \frac{1}{2^{l}l!}\Big(\frac{d}{dx}\Big)^{l}\Big(2l(x^{2} - 1)^{l-1}+(2l+1)(x^{2} - 1)^{l}\Big)
\end{align*}

Now, if the equation is expanded, certain terms will look familiar,

$$
P_{l+1}(x) = \frac{1}{2^{l}l!}\Big(\frac{d}{dx}\Big)^{l} 2l(x^{2} - 1)^{l-1} + \frac{1}{2^{l}l!}\Big(\frac{d}{dx}\Big)^{l}(2l+1)(x^{2} - 1)^{l}
$$

Breaking the RHS of the equation up into two components it is clear that

\begin{align*}
    P_{l-1}(x) &= \frac{1}{2^{l}l!}\Big(\frac{d}{dx}\Big)^{l} 2l(x^{2} - 1)^{l-1} & (2l+1)P_{l}(x) &= \frac{1}{2^{l}l!}\Big(\frac{d}{dx}\Big)^{l}(x^{2} - 1)^{l}
\end{align*}

Thus, the equation in its final form is

$$
\frac{dP_{l+1}}{dx} - \frac{dP_{l-1}}{dx} - (2l+1)P_{l} = 0
$$

\clearpage

% -------------------------- EOD -------------------------- 
\end{document}