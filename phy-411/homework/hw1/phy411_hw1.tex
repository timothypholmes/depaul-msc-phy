\documentclass[11pt]{article}

 \usepackage{tikz}
 \usepackage{tkz-euclide}
 \usepackage{amsthm,amssymb}

\newcommand{\HWnum}{1} 
\newcommand{\StudName}{Timothy Holmes} % author
\newcommand{\CourseNum}{411}           % course number
\newcommand{\Subject}{PHY}

\usepackage{graphicx, amsmath, amssymb,fancyhdr}
\addtolength{\textwidth}{1.5in}
\addtolength{\oddsidemargin}{-2cm}
\addtolength{\evensidemargin}{-2cm}
\addtolength{\textheight}{1.6in}
\addtolength{\topmargin}{-0.7in}
\addtolength{\headsep}{-0.1in}
%\addtolength{\footskip}{-0.2in}
\pagestyle{fancy}
\cfoot{}
\lhead{\textbf{\Subject~\CourseNum~--- Homework~\HWnum}}
\rhead{\textbf{\StudName:~Page~\thepage}}

\addtolength{\parskip}{\baselineskip} % skips a line between paragraphs
\parindent 0in                        % no indent at start of paragraph

\begin{document}
% -------------------------- BOD -------------------------- 

\title{Homework {\HWnum}}
\author{Timothy Holmes \\ \Subject~411 Electrodynamics I}

\maketitle

\section*{Problem 1}

\begin{center}
\begin{tikzpicture}

\tkzDefPoint(2,3.5){A}
\tkzDefPoint(0,0){B}
\tkzDefPoint(4,0){C}

\tkzDrawSegments(A,B B,C A,C)
\tkzLabelPoints[above,yshift=0pt](A)
\tkzLabelPoints[left,yshift=4pt](B)
\tkzLabelPoints[right,yshift=4pt](C)

\tkzDefMidPoint(A,B) \tkzGetPoint{D}
\tkzDefLine[orthogonal=through D](A,B) 

\tkzDefMidPoint(A,C) \tkzGetPoint{E}
\tkzDefLine[orthogonal=through E](A,C) 

\tkzDrawSegment(D,E)
\tkzLabelPoints[above,xshift=-2mm](D)
\tkzLabelPoints[above,xshift=2mm](E)
\end{tikzpicture}
\end{center}

\begin{proof}
Given the triangle above, let the line segment $\overrightarrow{AB}$ be $\vec{a}$ and let the line segment $\overrightarrow{AC}$ be $\vec{b}$. Since the midpoints are in the middle of $\overrightarrow{AB}$ and $\overrightarrow{AC}$, these line segments are half way to the midpoint. The line segments to the midpoint are defined as $\overrightarrow{AD}$ and $\overrightarrow{AE}$. Therefore, since  $\overrightarrow{AD}$  is half of $\overrightarrow{AB}$ and $\overrightarrow{AE}$  is half of $\overrightarrow{AC}$ then 

$$
\overrightarrow{AD} = \frac{1}{2}\vec{a}, ~~~~ \overrightarrow{AE} = \frac{1}{2}\vec{b}.
$$

The line segment $\overrightarrow{BC}$ as a vector is $\vec{b} - \vec{a}$. The line segment $\overrightarrow{DE}$ as a vector is $\frac{1}{2}(\vec{b} - \vec{a})$. The relation these line segments have is

$$
\overrightarrow{DE} = \frac{1}{2}(\vec{b} - \vec{a}) = \frac{1}{2} \overrightarrow{BC}.
$$

Thus, the mid line segment is half of the lower line segment or $\overrightarrow{DE} = \frac{1}{2}\overrightarrow{BC}$.

\end{proof}

% If you insert the command below
\clearpage

\section*{Problem 2}

Prove 
$$
\vec{\nabla} \times (\vec{\nabla} \times \vec{A}) 
= \vec{\nabla}(\vec{\nabla} \cdot \vec{A}) - \nabla^{2}\vec{A}
$$

To prove that the RHS is equivalent to the LHS using explicit calculation, the first calculation is then

$$
\vec{\nabla} \times \vec{A} = \Big( \hat{x}\Big(\frac{\partial}{\partial x} \Big) + \hat{y}\Big(\frac{\partial}{\partial y} \Big) + \hat{z}\Big(\frac{\partial}{\partial z} \Big)\Big) \times (A_{x}\hat{x} + A_{x}\hat{y} + A_{x}\hat{z})
$$

Taking the curl from above gives

\begin{eqnarray*}
= ( 
\hat{x}\frac{\partial}{\partial x} \times \vec{A}_{x}\hat{x} + \hat{x}\frac{\partial}{\partial x} \times \vec{A}_{y}\hat{y} + \hat{x}\frac{\partial}{\partial x} \times \vec{A}_{z}\hat{z} \\
\hat{y}\frac{\partial}{\partial y} \times \vec{A}_{x}\hat{x} + \hat{y}\frac{\partial}{\partial y} \times \vec{A}_{y}\hat{y} + \hat{y}\frac{\partial}{\partial y} \times \vec{A}_{z}\hat{z} \\
\hat{z}\frac{\partial}{\partial z}  \times \vec{A}_{x}\hat{x} + \hat{z}\frac{\partial}{\partial z}  \times \vec{A}_{y}\hat{y} + \hat{z}\frac{\partial}{\partial z}  \times \vec{A}_{z}\hat{z} 
) 
\end{eqnarray*}

and all of the same components will be zero. This can be simplified to 

\begin{eqnarray*}
A_{y}\frac{\partial}{\partial x}\hat{z} + A_{z}\frac{\partial}{\partial x}\hat{y} - A_{x}\frac{\partial}{\partial y}\hat{z} + A_{z}\frac{\partial}{\partial y}\hat{x} + A_{x}\frac{\partial}{\partial z}\hat{y} + A_{y}\frac{\partial}{\partial z}\hat{x}
\end{eqnarray*}

and further simplified to 

\begin{eqnarray*}
\Big(\frac{\partial A_{x}}{\partial y} - \frac{\partial A_{y}}{\partial z}\Big) \hat{x} + \Big(\frac{\partial A_{x}}{\partial z} - \frac{\partial A_{z}}{\partial x}\Big) \hat{y} +
\Big(\frac{\partial A_{y}}{\partial x} - \frac{\partial A_{x}}{\partial y}\Big) \hat{z}.
\end{eqnarray*}

Taking the curl one more time gives

$$
\vec{\nabla} \times (\vec{\nabla} \times \vec{A}) = 
$$

$$
\Big(\frac{\partial^{2}A_{y}}{\partial x \partial y} - \frac{\partial^{2} A_{x}}{\partial y^{2}} - \frac{\partial^{2} A_{x}}{\partial z^{2}} + \frac{\partial^{2} A_{z}}{\partial z \partial x} \big) \hat{x} \\
- \Big(-\frac{\partial^{2}A_{x}}{\partial y \partial x} + \frac{\partial^{2} A_{y}}{\partial x^{2}} + \frac{\partial^{2} A_{y}}{\partial z^{2}} - \frac{\partial^{2} A_{z}}{\partial y \partial z} \big) \hat{y} \\
+ \Big(\frac{\partial^{2}A_{x}}{\partial z \partial x} - \frac{\partial^{2} A_{z}}{\partial x^{2}} - \frac{\partial^{2} A_{z}}{\partial y^{2}} + \frac{\partial^{2} A_{y}}{\partial z \partial y} \big) \hat{z}
$$

Now, for the RHS starting with 

$$
\vec{\nabla}(\vec{\nabla} \cdot \vec{A}) - \nabla^{2}\vec{A} =
$$

\begin{center}
\begin{eqnarray*}
 = \Big(\frac{\partial}{\partial x}\hat{x} + \frac{\partial}{\partial y}\hat{y} + \frac{\partial}{\partial z}\hat{z}\Big)\Big(\frac{\partial A_{x}}{\partial x} + \frac{\partial A_{y}}{\partial y} + \frac{\partial A_{z}}{\partial z} \Big) - \Big(\frac{\partial^{2}}{\partial x^{2}} +\frac{\partial^{2}}{\partial y^{2}} + \frac{\partial^{2}}{\partial z^{2}}\Big) (A_{x}\hat{x} + A_{y}\hat{y} + A_{z}\hat{z})
\end{eqnarray*}
\end{center}

\begin{center}
\begin{eqnarray*}
= \Big(\frac{\partial^{2}A_{y}}{\partial y \partial x} - \frac{\partial^{2} A_{x}}{\partial y^{2}} - \frac{\partial^{2} A_{x}}{\partial z^{2}} + \frac{\partial^{2} A_{z}}{\partial z \partial x} \big) \hat{x} 
+ \Big(\frac{\partial^{2}A_{x}}{\partial x \partial y} - \frac{\partial^{2} A_{y}}{\partial y^{x}} - \frac{\partial^{2} A_{y}}{\partial z^{2}} + \frac{\partial^{2} A_{z}}{\partial z \partial y} \big) \hat{y}
+ \Big(\frac{\partial^{2}A_{x}}{\partial x \partial z} - \frac{\partial^{2} A_{z}}{\partial x^{2}} - \frac{\partial^{2} A_{z}}{\partial y^{2}} + \frac{\partial^{2} A_{y}}{\partial y \partial z} \big) \hat{z}
\end{eqnarray*}
\end{center}

Thus, because the LHS is equal to the RHS using explicit calculation then $\vec{\nabla} \times (\vec{\nabla} \times \vec{A}) 
= \vec{\nabla}(\vec{\nabla} \cdot \vec{A}) - \nabla^{2}\vec{A}$ is true.

\clearpage

\section*{Problem 3}

Prove 
$$
\vec{A} \cdot (\vec{B} \times \vec{C}) = 
\vec{B} \cdot (\vec{C} \times \vec{A}) = 
\vec{C} \cdot (\vec{A} \times \vec{B}) 
$$

To prove that all combinations of calculations above are the same with
the method of explicit calculations, the calculations need to be broke up. 
Starting with the cross product in the parenthesis. 

\begin{eqnarray*}
\vec{B} \times \vec{C} = 
(B_{x}\hat{x} + \hat{y}\frac{\partial}{\partial y} + B_{z}\hat{z}) \times
(B_{x}\hat{x} + B_{y}\hat{y} + B_{z}\hat{z}) 
\end{eqnarray*}

\begin{eqnarray*}
= ( 
B_{x}\hat{x} \times C_{x}\hat{x} + B_{x}\hat{x} \times C_{y}\hat{y} + B_{x}\hat{x} \times C_{z}\hat{z} \\
B_{y}\hat{y} \times C_{x}\hat{x} + B_{y}\hat{y} \times C_{y}\hat{y} + B_{y}\hat{y} \times C_{z}\hat{z} \\
B_{z}\hat{z} \times C_{x}\hat{x} + B_{z}\hat{z} \times C_{y}\hat{y} + B_{z}\hat{z} \times C_{z}\hat{z} 
) 
\end{eqnarray*}

\begin{eqnarray*}
= B_{x}C_{y}\hat{z} - B_{x}C{z}\hat{y} 
- B_{y}C_{x}\hat{z} + B_{y}C_{z}\hat{x} 
+ B_{z}C_{x}\hat{y} - B_{z}C_{y}\hat{x}
\end{eqnarray*}

This can be simplified by factoring out the vector components of similar direction,

\begin{eqnarray*}
= (B_{y}C_{z} - B_{z}C_{y})\hat{x} 
+ (B_{y}C_{x} - B_{z}C_{x})\hat{y}
+ (B_{x}C_{y} - B_{z}C_{x})\hat{z}.
\end{eqnarray*}

Now, the dot product of $\vec{A}$ on the above equation will satisfy 
the calculation for this combination of vectors. So 

$$
\vec{A} \cdot (\vec{B} \times \vec{C})
$$

\begin{eqnarray*}
= (A_{x}\hat{x} + A_{y}\hat{y} + A_{z}\hat{z}) \cdot
(B_{y}C_{z} - B_{z}C_{y})\hat{x} 
+ (B_{y}C_{x} - B_{z}C_{x})\hat{y}
+ (B_{x}C_{y} - B_{z}C_{x})\hat{z}.
\end{eqnarray*}

Taking the dot product such that $\vec{A} \cdot (\vec{B} \times \vec{C})$
results as
    
\begin{eqnarray*}
= A_{x}B_{y}C_{z} - A_{x}B_{z}C_{y} + A_{y}B_{y}C_{x} 
- A_{y}B_{z}C_{x} + A_{z}B_{x}C_{y} - A_{z}B_{z}C_{x}
\end{eqnarray*}

% 2-----------------

The same steps as above can be done for

\begin{eqnarray*}
\vec{C} \times \vec{A} = 
(C_{x}\hat{x} + C_{y}\hat{y} + C_{z}\hat{z}) \times
(C_{x}\hat{x} + C_{y}\hat{y} + C_{z}\hat{z}) 
\end{eqnarray*}

\begin{eqnarray*}
= ( 
C_{x}\hat{x} \times A_{x}\hat{x} + C_{x}\hat{x} \times A_{y}\hat{y} + C_{x}\hat{x} \times A_{z}\hat{z} \\
C_{y}\hat{y} \times A_{x}\hat{x} + C_{y}\hat{y} \times A_{y}\hat{y} + C_{y}\hat{y} \times A_{z}\hat{z} \\
C_{z}\hat{z} \times A_{x}\hat{x} + C_{z}\hat{z} \times A_{y}\hat{y} + C_{z}\hat{z} \times A_{z}\hat{z} 
) 
\end{eqnarray*}

\begin{eqnarray*}
= C_{x}A_{y}\hat{z} - C_{x}A{z}\hat{y} 
- C_{y}A_{x}\hat{z} + C_{y}A_{z}\hat{x} 
+ C_{z}A_{x}\hat{y} - C_{z}A_{y}\hat{x}
\end{eqnarray*}

This can be simplified by factoring out the vector components of similar direction,

\begin{eqnarray*}
= (C_{y}A_{z} - C_{z}A_{y})\hat{x} 
+ (C_{y}A_{x} - C_{z}A_{x})\hat{y}
+ (C_{x}A_{y} - C_{z}A_{x})\hat{z}.
\end{eqnarray*}

Now, the dot product of $\vec{A}$ on the above equation will satisfy 
the calculation for this combination of vectors. So 

$$
\vec{B} \cdot (\vec{C} \times \vec{A}) 
$$

\begin{eqnarray*}
= (B_{x}\hat{x} + B_{y}\hat{y} + B_{z}\hat{z}) \cdot
(C_{y}A_{z} - C_{z}A_{y})\hat{x} 
+ (C_{y}A_{x} - C_{z}A_{x})\hat{y}
+ (C_{x}A_{y} - C_{z}A_{x})\hat{z}.
\end{eqnarray*}

Taking the dot product such that $\vec{A} \cdot (\vec{C} \times \vec{A})$
results as
    
\begin{eqnarray*}
= B_{x}C_{y}A_{z} - B_{x}C_{z}A_{y} + B_{y}C_{y}A_{x} 
- B_{y}C_{z}A_{x} + B_{z}C_{x}A_{y} - B_{z}C_{z}A_{x}
\end{eqnarray*}


Taking the dot product such that $\vec{A} \cdot (\vec{B} \times \vec{C})$
results as
    
\begin{eqnarray*}
= A_{z}B_{x}C_{y} - A_{y}B_{x}C_{z} + A_{z}B_{y}C_{x} + A_{y}B_{z}C_{x} - A_{x}B_{z}C_{y}
\end{eqnarray*}

% 3-----------------

Finally

\begin{eqnarray*}
\vec{A} \times \vec{B} = 
(A_{x}\hat{x} + A_{y}\hat{y} + A_{z}\hat{z}) \times
(B_{x}\hat{x} + B_{y}\hat{y} + B_{z}\hat{z}) 
\end{eqnarray*}

\begin{eqnarray*}
= ( 
A_{x}\hat{x} \times B_{x}\hat{x} + A_{x}\hat{x} \times B_{y}\hat{y} + A_{x}\hat{x} \times B_{z}\hat{z} \\
A_{y}\hat{y} \times B_{x}\hat{x} + A_{y}\hat{y} \times B_{y}\hat{y} + A_{y}\hat{y} \times B_{z}\hat{z} \\
A_{z}\hat{z} \times B_{x}\hat{x} + A_{z}\hat{z} \times B_{y}\hat{y} + A_{z}\hat{z} \times B_{z}\hat{z} 
) 
\end{eqnarray*}

\begin{eqnarray*}
= A_{x}B_{y}\hat{z} - A_{x}B{z}\hat{y} 
- A_{y}B_{x}\hat{z} + A_{y}B_{z}\hat{x} 
+ A_{z}B_{x}\hat{y} - A_{z}B_{y}\hat{x}
\end{eqnarray*}

This can be simplified by factoring out the vector components of similar direction,

\begin{eqnarray*}
= (A_{y}B_{z} - A_{z}B_{y})\hat{x} 
+ (A_{y}B_{x} - A_{z}B_{x})\hat{y}
+ (A_{x}B_{y} - A_{z}B_{x})\hat{z}.
\end{eqnarray*}

Now, the dot product of $\vec{A}$ on the above equation will satisfy 
the calculation for this combination of vectors. So 

$$
\vec{C} \cdot (\vec{A} \times \vec{B}) 
$$

\begin{eqnarray*}
= (A_{x}\hat{x} + A_{y}\hat{y} + A_{z}\hat{z}) \cdot
(A_{y}B_{z} - A_{z}B_{y})\hat{x} 
+ (A_{y}B_{x} - A_{z}B_{x})\hat{y}
+ (A_{x}B_{y} - A_{z}B_{x})\hat{z}.
\end{eqnarray*}

Taking the dot product such that $\vec{C} \cdot (\vec{A} \times \vec{B})$
results as
    
\begin{eqnarray*}
= A_{y}B_{z}C_{x} - A_{z}B_{y}C_{x} + A_{z}B_{x}C_{y} 
- A_{x}B_{z}C_{y} + A_{x}B_{y}C_{z} - A_{y}B_{x}C_{z}.
\end{eqnarray*}

Each result had a different order, but ordering in the same way shows that they all have the exact same components which proves that $\vec{A} \cdot (\vec{B} \times \vec{C}) = \vec{B} \cdot (\vec{C} \times \vec{A}) = \vec{C} \cdot (\vec{A} \times \vec{B})$.


\clearpage

\section*{Problem 4}

The integral to integrate over all space using spherical coordinates is found using the following
$$
Q = \int_{0}^{2\pi} \int_{0}^{\pi} \int_{0}^{\infty} \rho(r, \theta, \phi) r^{2}sin(\theta) d\phi d\theta dr.
$$

Substituting $\rho(x)$ in for the Dirac $\delta$-function, $C\delta(r - R)$, creates the integral 

$$
Q = \int_{0}^{2\pi} \int_{0}^{\pi} \int_{0}^{\infty}  C\delta(r - R) r^{2}sin(\theta) d\phi d\theta dr.
$$

First integrating over $\phi$ gives

$$
Q =  2\pi \int_{0}^{\pi} \int_{0}^{\infty}  C\delta(r - R) r^{2}sin(\theta) d\theta dr .
$$

Then integrating over $\theta$ gives

$$
Q =  4\pi \int_{0}^{\infty}  C\delta(r - R) r^{2} dr.
$$

Finally integrating over $r$ gives

$$
Q =  4\pi C \int_{0}^{\infty} \delta(r - R) r^{2} dr.
$$

The selector function works in a way as such

$$
\int_{0}^{\infty} f(a)\delta(a - A) = A.
$$

Therefore, for this problem the function defined as $f(a)$ in the integral above is $r^{2}$. The Delta function is $\delta(r - R)$, therefore, $r$ is going to be replaced with $R$. Then, integrating over all space the total charge is now

$$
Q =  \frac{4\pi}{C R^{2}}.
$$

The constant can now be solved for and becomes

$$
C =  \frac{Q}{4\pi  R^{2}}.
$$

% -------------------------- EOD -------------------------- 
\end{document}