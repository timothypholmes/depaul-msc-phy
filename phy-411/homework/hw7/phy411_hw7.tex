\documentclass[11pt]{article}

\newcommand{\HWnum}{7} 
\newcommand{\StudName}{Timothy Holmes} % author
\newcommand{\CourseNum}{411}           % course number
\newcommand{\Subject}{PHY}

\usepackage{graphicx, amsmath, amssymb,fancyhdr}
\addtolength{\textwidth}{1.5in}
\addtolength{\oddsidemargin}{-2cm}
\addtolength{\evensidemargin}{-2cm}
\addtolength{\textheight}{1.6in}
\addtolength{\topmargin}{-0.7in}
\addtolength{\headsep}{-0.1in}
%\addtolength{\footskip}{-0.2in}
\pagestyle{fancy}
\cfoot{}
\lhead{\textbf{\Subject~\CourseNum~--- Homework~\HWnum}}
\rhead{\textbf{\StudName:~Page~\thepage}}

\addtolength{\parskip}{\baselineskip} % skips a line between paragraphs
\parindent 0in                        % no indent at start of paragraph

\newcommand{\dd}{\textrm{d}}
\usepackage{braket}
\usepackage{lipsum, babel}
\usepackage{blindtext}
\usepackage{tikz}

\begin{document}
% -------------------------- BOD -------------------------- 

\title{Homework {\HWnum}}
\author{Timothy Holmes \\ \Subject~411 Electrodynamics I}

\maketitle

\section*{Problem 1}

The potential of a localized distribution of charge described by the charge density $\rho(\vec{x})$ is given by 

$$
\Phi(\vec{x}) = \frac{1}{4\pi\epsilon_{0}} \sum_{l=0}^{\infty} \sum_{m=-l}^{l} \frac{4\pi}{2l+1}q_{lm}\frac{Y_{lm}(\theta, \phi)}{r^{l+1}}
$$

where the \textit{multipole moments} $q_{lm}$ are given by

$$
q_{lm} = \int Y^{*}_{lm}(\theta', \phi')r'^{l} \rho(\vec{x}) d^{3}x'.
$$

The evaluation for $q_{11}$ means that $l = 1$ and $m = 1$. Therefore, starting with the equation gave above we have

$$
q_{11} = \int Y^{*}_{11}(\theta', \phi') r'^{1} \rho(\vec{x}) d^{3}x'
$$

where $Y^{*}_{11}$ is 

\begin{align*}
Y^{*}_{lm} &= (-1)^{m} \sqrt{\frac{2l+1}{4\pi}\frac{(l-m)!}{(l+m)!}} P_{l}^{m}(cos\theta')e^{-im\phi'} \\
Y^{*}_{11} &= -\sqrt{\frac{3}{4\pi}\frac{1}{2}} (cos\theta')e^{-i\phi'} \\ 
Y^{*}_{11}&= -\sqrt{\frac{3}{8\pi}} (cos\theta'cos(\phi') - icos\theta'sin(\phi'))
\end{align*}

Entering this back into the equation $q_{11}$ will produce the equation

\begin{align*}
q_{11} = -\sqrt{\frac{3}{8\pi}} \int \rho(\vec{x})((r'cos\theta')cos(\phi') - i(r'cos\theta')sin(\phi')) d^{3}x'
\end{align*}

From Spherical coordinates it is know that to Cartesian coordinates it is known that 

\begin{align*}
x &= rsin\theta cos\phi & y &= rsin\theta sin\phi & z &= rcos\theta
\end{align*}

Therefore, rewriting the problem in Cartesian coordinates the equation becomes

$$
q_{11} = -\sqrt{\frac{3}{8\pi}} \int \rho(\vec{x})((x - iy) d^{3}x'.
$$

Now, $p$ is given by

$$
\vec{p} = \int \vec{x}'d^{3}x'
$$

Thus, the final equation is

$$
q_{11} = -\sqrt{\frac{3}{8\pi}}(p_{x} - ip_{y}).
$$

The evaluation for $q_{10}$ means that $l = 1$ and $m = 0$. Therefore, starting with the equation gave above we have

$$
q_{10} = \int Y^{*}_{10}(\theta', \phi') r'^{1} \rho(\vec{x}) d^{3}x'
$$

where $Y^{*}_{10}$ is 

\begin{align*}
Y^{*}_{lm} &= (-1)^{m} \sqrt{\frac{2l+1}{4\pi}\frac{(l-m)!}{(l+m)!}} P_{l}^{m}(cos\theta')e^{-im\phi'} \\
Y^{*}_{10} &= \sqrt{\frac{3}{4\pi}\frac{1}{1}} (cos\theta') \\ 
Y^{*}_{10} &= \sqrt{\frac{3}{4\pi}} (cos\theta') 
\end{align*}

Entering this all back in gives 

$$
q_{10} = \sqrt{\frac{3}{4\pi}} \int \rho(\vec{x}) r' cos\theta' d^{3}x'.
$$

Therefore, rewriting the problem in Cartesian coordinates the equation becomes

$$
q_{10} = \sqrt{\frac{3}{4\pi}} \int \rho(\vec{x}) z' d^{3}x'.
$$

With our relation to what $p$ is the final equation becomes

$$
q_{10} = \sqrt{\frac{3}{4\pi}}p_{z}.
$$


\clearpage

\section*{Problem 2}

Show that 

$$
q_{21} = -\frac{1}{3}\sqrt{\frac{15}{8\pi}}(Q_{13}-iQ_{23})
$$

where $Q_{ij}$ is the quadrupole moment tensor given by

$$
Q_{ij} = \int (3x'_{i}x'_{j}-r'^{2}\delta_{ij})\rho(\vec{x'})d^{3}x'.
$$

The evaluation for $q_{21}$ means that $l = 2$ and $m = 1$. Therefore, starting with the equation gave above we have

$$
q_{21} = \int Y^{*}_{21}(\theta', \phi') r'^{2} \rho(\vec{x}) d^{3}x'
$$

where $Y^{*}_{21}$ is 

\begin{align*}
Y^{*}_{lm} &= (-1)^{m} \sqrt{\frac{2l+1}{4\pi}\frac{(l-m)!}{(l+m)!}} P_{l}^{m}(cos\theta')e^{-im\phi'} \\Y^{*}_{21} &= \sqrt{\frac{15}{8\pi}}sin(\theta')\cos(\theta')e^{-i\phi'}\\ 
&= \sqrt{\frac{15}{8\pi}}sin(\theta')(cos(\theta')cos(\phi') - icos(\theta')sin(\phi'))
\end{align*}

Entering this back into the main equation gives

$$
q_{21} = \sqrt{\frac{15}{8\pi}}sin(\theta')(cos(\theta')cos(\phi') - icos(\theta')sin(\phi'))r'^{2} \rho(\vec{x}) d^{3}x'
$$

Converting from spherical coordinates gives 

$$
q_{21} = \sqrt{\frac{15}{8\pi}} (x'z' - iy'z') \rho(\vec{x}) d^{3}x'
$$

Where

\begin{align*}
Q_{13} &= \int (3x'z') \rho(\vec{x}) d^{3}x' & Q_{23} &= \int (3y'z') \rho(\vec{x}) d^{3}x'
\end{align*}

Divide a 3 out for the equations above and the final equation is

$$
q_{21} = -\frac{1}{3}\sqrt{\frac{15}{8\pi}}(Q_{13}-iQ_{23}).
$$

\clearpage

\section*{Problem 3}

\begin{align*}
    E_{r} &= \frac{(l+1)}{(2l+1)\epsilon_{0}}q_{lm} \frac{Y_{lm}(\theta, \phi)}{r^{l+2}} &
    E_{\theta} &= -\frac{1}{(2l+1)\epsilon_{0}} q_{lm} \frac{1}{r^{l+2}}\frac{\partial}{\partial \theta} Y_{lm}(\theta, \phi) &
    E_{\phi} &= \frac{1}{(2l+1)\epsilon_{0}}q_{lm} \frac{1}{r^{l+2}} \frac{im}{sin\theta}Y_{lm}(\theta, \phi)
\end{align*}

For a dipole $\vec{p}$ along the z-axis, show that the fields above reduce to:

\begin{align*}
    E_{r} &= \frac{2pcos\theta}{4\pi\epsilon_{0}r^{3}} &
    E_{\theta} &= -\frac{psin\theta}{4\pi\epsilon_{0}r^{3}}&
    E_{\phi} &= 0
\end{align*}

The dipole $\vec{p}$ is given along the $z$-axis. From problem 1 we found that 

$$
q_{11} = -\sqrt{\frac{3}{8\pi}}(p_{x} - ip_{y}).
$$


But since this dipole is along the $z$-axis that means that $p_{x} = 0$ and $p_{y}=0$. With both $p_{x} = 0$ and $p_{y}=0$ this means that $q_{11} = 0$. On the other hand, from problem 1 we also found that

$$
q_{10} = \sqrt{\frac{3}{4\pi}}p_{z}.
$$

Therefore, this is the only $q$ value that will survive. More importantly this means that $l = 1$ and $m = 0$. From Jackson

$$
Y_{10} = \sqrt{\frac{3}{4\pi}} cos(\theta)
$$

Substituting all of this into the equation $E_{r}$ gives us

\begin{align*}
    E_{r} &= \frac{(1+1)}{(2(1)+1)\epsilon_{0}}q_{10} \frac{Y_{10}(\theta, \phi)}{r^{1+2}} \\
    &= \frac{2}{3\epsilon_{0}} \sqrt{\frac{3}{4\pi}}p_{z} \frac{1}{r^{3}} \sqrt{\frac{3}{4\pi}} cos(\theta) \\
\end{align*}

This reduces to 

$$
E_{r} &= \frac{2pcos\theta}{4\pi\epsilon_{0}r^{3}} 
$$

Applying the same process for $E_{\theta}$ will give

\begin{align*}
    E_{\theta} &= -\frac{1}{(2(1)+1)\epsilon_{0}}q_{10} \frac{1}{r^{1+2}}\frac{\partial}{\partial \theta} Y_{10}(\theta, \phi) \\
    &= -\frac{1}{3\epsilon_{0}} \sqrt{\frac{3}{4\pi}}p_{z} \frac{1}{r^{3}} \frac{\partial}{\partial \theta} \sqrt{\frac{3}{4\pi}} cos(\theta) 
\end{align*}

where 

$$
\frac{\partial}{\partial \theta} Y_{10}(\theta, \phi) = \frac{\partial}{\partial \theta} \sqrt{\frac{3}{4\pi}} cos(\theta) = -\sqrt{\frac{3}{4\pi}} sin(\theta).
$$

Therefore, the equation will reduce to

$$
E_{\theta} &= -\frac{psin\theta}{4\pi\epsilon_{0}r^{3}}.
$$

Finally, applying the same process for $E_{\phi}$ will give

\begin{align*}
    E_{\phi} &= \frac{1}{(2(1)+1)\epsilon_{0}}q_{10} \frac{1}{r^{1+2}} \frac{i(0)}{sin\theta}Y_{10}(\theta, \phi) \\
\end{align*}

Since $m = 0$ and there is a fraction in this equation that is $(im)/sin\theta$ the entire equation will be zero. Thus, 

$$
E_{\phi} &= 0.
$$

\clearpage

\section*{Problem 4}

Suppose that we have a uniform magnetic field $\vec{B_{0}} = B_{0}\hat{z}$, where $B_{0}$ is a constant.

\subsection*{(a)}

Examine whether

$$
\vec{A} = \frac{\vec{B_{0}}}{2} \times \vec{x} 
$$

is an appropriate vector potential for this given field.

The uniform magnetic field can be entered into the vector potential as

\begin{align*}
\vec{A} &= \frac{B_{0}\hat{z}}{2} \times \vec{x} \\
        &= \frac{\vec{B_{0}}\hat{z}}{2} \times (x\vec{x} + y\vec{y} + z\vec{z}) \\
        &= \frac{\vec{B_{0}}}{2} (x\hat{y} - y\hat{x})
\end{align*}

The curl of $\vec{A}$ has to be zero for this to be an appropriate vector. Therefore, 

\begin{align*}
\vec{\nabla} \times \vec{A} &= \Big( \hat{x}\Big(\frac{\partial}{\partial x} \Big) +  \hat{y}\Big(\frac{\partial}{\partial y} \Big) +  \hat{z}\Big(\frac{\partial}{\partial z} \Big)\Big) \times \frac{\vec{B_{0}}}{2} (x\hat{y} - y\hat{x}) \\
&= \frac{\partial}{\partial x}x\hat{z} - \frac{\partial}{\partial z}x\hat{x} + \frac{\partial}{\partial y}y\hat{x} + \frac{\partial}{\partial z}y\hat{y} \\
&= \hat{z}+\hat{x}
\end{align*}

This will obviously not be zero so it is not an appropriate vector.

\subsection*{(b)} Does this vector potential satisfy the Coulomb gauge, $\vec{\nabla} \cdot \vec{A} = 0$?

Given

\begin{align*}
\vec{\nabla} \cdot \vec{A} &=  \Big( \hat{x}\Big(\frac{\partial}{\partial x} \Big) +  \hat{y}\Big(\frac{\partial}{\partial y} \Big) +  \hat{z}\Big(\frac{\partial}{\partial z} \Big)\Big) \cdot \frac{\vec{B_{0}}}{2} (x\hat{y} - y\hat{x})\\
&= \frac{\vec{B_{0}}}{2} (\frac{\partial}{\partial y}x - \frac{\partial}{\partial x}y)
\end{align*}

\begin{align*}
\vec{\nabla} \cdot \vec{A} &=  \Big( \hat{x}\Big(\frac{\partial}{\partial x} \Big) +  \hat{y}\Big(\frac{\partial}{\partial y} \Big) +  \hat{z}\Big(\frac{\partial}{\partial z} \Big)\Big) \cdot \frac{\vec{B_{0}}}{2} (x\hat{y} - y\hat{x})\\
&= \frac{\vec{B_{0}}}{2} (\frac{\partial}{\partial y}x - \frac{\partial}{\partial x}y)
\end{align*}

where 

\begin{align*}
\frac{\partial}{\partial y}x &= 0 & \frac{\partial}{\partial x}y &= 0
\end{align*}

so

$$
\vec{\nabla} \cdot \vec{A} &= 0
$$

\clearpage

% -------------------------- EOD -------------------------- 
\end{document}