% --------------------------------------------------------------
%                           Set Up
% --------------------------------------------------------------
 
\documentclass[12pt]{article}
 
\usepackage[margin=1in]{geometry} 
\usepackage{amsmath,amsthm,amssymb}
 
\newcommand{\N}{\mathbb{N}}
\newcommand{\Z}{\mathbb{Z}}
 
\newenvironment{theorem}[2][Theorem]{\begin{trivlist}
\item[\hskip \labelsep {\bfseries #1}\hskip \labelsep {\bfseries #2.}]}{\end{trivlist}}
\newenvironment{lemma}[2][Lemma]{\begin{trivlist}
\item[\hskip \labelsep {\bfseries #1}\hskip \labelsep {\bfseries #2.}]}{\end{trivlist}}
\newenvironment{exercise}[2][Exercise]{\begin{trivlist}
\item[\hskip \labelsep {\bfseries #1}\hskip \labelsep {\bfseries #2.}]}{\end{trivlist}}
\newenvironment{problem}[2][Problem]{\begin{trivlist}
\item[\hskip \labelsep {\bfseries #1}\hskip \labelsep {\bfseries #2.}]}{\end{trivlist}}
\newenvironment{question}[2][Question]{\begin{trivlist}
\item[\hskip \labelsep {\bfseries #1}\hskip \labelsep {\bfseries #2.}]}{\end{trivlist}}
\newenvironment{corollary}[2][Corollary]{\begin{trivlist}
\item[\hskip \labelsep {\bfseries #1}\hskip \labelsep {\bfseries #2.}]}{\end{trivlist}}

\newenvironment{solution}{\begin{proof}[Solution]}{\end{proof}}
 
\begin{document}
 
% --------------------------------------------------------------
%                         Start here
% --------------------------------------------------------------
 
\title{Quiz 8}
\author{Timothy Holmes\\ %replace with your name
Classical Mechanics}

\maketitle

\section*{1.}

I would like to know more about the angular momentum chapter. More specifically how the vector calculus works itself out.


\section*{2.}

For infinitesimal canonical transforms there are two sets of coordinates. They are the old coordinates and the new coordinates. These tend to be the lower case to upper case values such as q to Q. There is also then old momentum and new momentum. The coordinate transformation is is responsible for the 'Displacement' in time.

\section*{3.}

Liouville's theorem tells us the desneisity of points in the system the represent the particles. It is related to the hamiltonian since the equation needs the generalized coordinates, velocity, momentum, and acceleration from the system. 



% --------------------------------------------------------------
%                           End Document.
% --------------------------------------------------------------
 
\end{document}