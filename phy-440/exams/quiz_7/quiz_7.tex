% --------------------------------------------------------------
%                           Set Up
% --------------------------------------------------------------
 
\documentclass[12pt]{article}
 
\usepackage[margin=1in]{geometry} 
\usepackage{amsmath,amsthm,amssymb}
 
\newcommand{\N}{\mathbb{N}}
\newcommand{\Z}{\mathbb{Z}}
 
\newenvironment{theorem}[2][Theorem]{\begin{trivlist}
\item[\hskip \labelsep {\bfseries #1}\hskip \labelsep {\bfseries #2.}]}{\end{trivlist}}
\newenvironment{lemma}[2][Lemma]{\begin{trivlist}
\item[\hskip \labelsep {\bfseries #1}\hskip \labelsep {\bfseries #2.}]}{\end{trivlist}}
\newenvironment{exercise}[2][Exercise]{\begin{trivlist}
\item[\hskip \labelsep {\bfseries #1}\hskip \labelsep {\bfseries #2.}]}{\end{trivlist}}
\newenvironment{problem}[2][Problem]{\begin{trivlist}
\item[\hskip \labelsep {\bfseries #1}\hskip \labelsep {\bfseries #2.}]}{\end{trivlist}}
\newenvironment{question}[2][Question]{\begin{trivlist}
\item[\hskip \labelsep {\bfseries #1}\hskip \labelsep {\bfseries #2.}]}{\end{trivlist}}
\newenvironment{corollary}[2][Corollary]{\begin{trivlist}
\item[\hskip \labelsep {\bfseries #1}\hskip \labelsep {\bfseries #2.}]}{\end{trivlist}}

\newenvironment{solution}{\begin{proof}[Solution]}{\end{proof}}
 
\begin{document}
 
% --------------------------------------------------------------
%                         Start here
% --------------------------------------------------------------
 
\title{Quiz 7}
\author{Timothy Holmes\\ %replace with your name
Classical Mechanics}

\maketitle

\section*{1.}

Why is it that we can keep the coordinates the same and achieve a similar result by just taking derivatives?


\section*{2.}
What's the difference between a point transformation and a canonical transformation? Are
there restrictions on one type of transformation that are not present in the other? \\

Point transformations take place in configuration space, where it would yield new space axes. A canonical transformation is a transformation that happens in phase space. A phase space transformation is one that requires a partial derivative of the Hamiltonian with respect to the momenta $p_{i}$ and velocity $q_{i}$.

\section*{3.}
What is a generating function? Where does it come from and how is it used? \\

A generating function a partial derivative that creates a equation of motion for a system. These come from the canonical partial derivatives of the Hamiltonian. This also is used for integrating like we did for Lagrangians where we integrated the equation of motion twice to get q(t).
% --------------------------------------------------------------
%                           End Document.
% --------------------------------------------------------------
 
\end{document}