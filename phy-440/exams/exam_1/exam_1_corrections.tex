% --------------------------------------------------------------
%                           Set Up
% --------------------------------------------------------------
 
\documentclass[12pt]{article}
 
\usepackage[margin=1in]{geometry} 
\usepackage{amsmath,amsthm,amssymb}
 
\newcommand{\N}{\mathbb{N}}
\newcommand{\Z}{\mathbb{Z}}
 
\newenvironment{theorem}[2][Theorem]{\begin{trivlist}
\item[\hskip \labelsep {\bfseries #1}\hskip \labelsep {\bfseries #2.}]}{\end{trivlist}}
\newenvironment{lemma}[2][Lemma]{\begin{trivlist}
\item[\hskip \labelsep {\bfseries #1}\hskip \labelsep {\bfseries #2.}]}{\end{trivlist}}
\newenvironment{exercise}[2][Exercise]{\begin{trivlist}
\item[\hskip \labelsep {\bfseries #1}\hskip \labelsep {\bfseries #2.}]}{\end{trivlist}}
\newenvironment{problem}[2][Problem]{\begin{trivlist}
\item[\hskip \labelsep {\bfseries #1}\hskip \labelsep {\bfseries #2.}]}{\end{trivlist}}
\newenvironment{question}[2][Question]{\begin{trivlist}
\item[\hskip \labelsep {\bfseries #1}\hskip \labelsep {\bfseries #2.}]}{\end{trivlist}}
\newenvironment{corollary}[2][Corollary]{\begin{trivlist}
\item[\hskip \labelsep {\bfseries #1}\hskip \labelsep {\bfseries #2.}]}{\end{trivlist}}

\newenvironment{solution}{\begin{proof}[Solution]}{\end{proof}}
 
\begin{document}
 
% --------------------------------------------------------------
%                         Start here
% --------------------------------------------------------------
 
\title{Exam 1 Corrections}
\author{Timothy Holmes\\ %replace with your name
PHY 440 Classical Mechanics}

\maketitle

\section*{Problem 1}
\noindent
On the exam I completely misread/misinterpreted this problem. I read this problem as find the constrains for whatever reason. Looking back at this problem I can now clearly solve these problems. To find these we just take the Euler-Lagrange equation

$$
\frac{d}{dt} \frac{\partial L}{\partial \dot{q}} - \frac{\partial L}{q} = 0.
$$

\subsection*{1a.}
\noindent
Given,

$$
L = \frac{1}{2}m(\dot{x}^{2} + \dot{y}^{2} + \dot{z}^{2})
$$
\noindent
After the derivatives there are 3 const. motions. 

$$
\frac{d}{dt} \frac{\partial L}{\partial \dot{x}} - \frac{\partial L}{\partial x} = 0 \rightarrow \ddot{x}m = const.
$$

$$
\frac{d}{dt} \frac{\partial L}{\partial \dot{y}} - \frac{\partial L}{\partial y} = 0 \rightarrow \ddot{y}m = const.
$$

$$
\frac{d}{dt} \frac{\partial L}{\partial \dot{z}} - \frac{\partial L}{\partial z} = 0 \rightarrow \ddot{z}m = const.
$$

\subsection*{1c.}
\noindent
Given,

$$
L = \frac{1}{2}m\dot{x}^{2} + \frac{1}{2}I\dot{\theta}^{2} - mg(l - x)sin(\alpha)
$$

$x$ will becomes $x = \theta R$ for this transformation. 

$$
\frac{d}{dt} \frac{\partial L}{\partial \dot{x}} - \frac{\partial L}{\partial x} = 0 \rightarrow \ddot{x}m + mgsin(\alpha)
$$
$$
\frac{d}{dt} \frac{\partial L}{\partial \dot{\theta}} - \frac{\partial L}{\partial \theta} = 0 \rightarrow m\ddot{\theta}R^{2} + I\ddot{\theta} = mgRsin(\alpha) 
$$
\noindent
$\theta$ is not a constant of motion for this Lagrangian since the momentum in not conserved. 


\subsection*{1d.}
\noindent
Given,

$$
L = \frac{1}{2}ml^{2}(\dot{\theta}^{2} + \dot{\phi}^{2}sin^{2}\theta) + mglcos(\theta)
$$
\noindent
There will not be a const. of motion for $\theta$ since there is a $\dot{\theta}$ and ${\theta}$.

$$
\frac{d}{dt} \frac{\partial L}{\partial \dot{\phi}} - \frac{\partial L}{\partial \phi} = 0 \rightarrow \ddot{\phi}ml^{2}sin^{2}\theta = const.
$$



\subsection*{1e.}
\noindent
Given,

$$
L = \frac{1}{2}m(\dot{r}^{2} + r^{2}\dot{\theta}^{2}) + \frac{GMm}{r}
$$
\noindent
since there is a $\dot{r}$ and $r$ there will not be an const. of motion.

$$
\frac{d}{dt} \frac{\partial L}{\partial \dot{\phi}} - \frac{\partial L}{\partial \phi} = 0 \rightarrow \ddot{\phi}ml^{2}sin^{2}\theta = const.
$$

\section*{Problem 2}


\subsection*{2a.}

I was initial confused on this problem because I was unsure if the block could move or not. For the corrections I will assume that the wedge does move. For this problem we also need to know the distance the block is from the top of the wedge. I had erased this even though it was correct. The wedge of mass $m_{1}$ can only move in the x direction so it is 

$$
T = \frac{1}{2}m_{1} \dot{x_{1}^{2}}
$$

and the smaller mass of $m_{2}$ is 

$$
T = \frac{1}{2}m_{2}(\dot{x_{2}^{2} + \dot{y_{2}}^{2}})
$$

Transforming the coordinates brings 

\begin{align}
x_{2} &= lcos(\alpha)  &\dot{x}^{2} = \dot{x_{1}} + \dot{l}cos{\alpha} \\
y_{2} &= h - lsin(\alpha) & \dot{y_{2}} = -\dot{l}sin(\alpha) \\
y_{1} &= 0 
\end{align}

where l is the distance from the block to the top of the wedge and h is the height of the wedge. 

$$
T = \frac{1}{2}m_{1}\dot{x}^{2} + \frac{1}{2}m_{2}(2\dot{x}\dot{l}cos(\alpha) + \dot{l}^{2} + \dot{x}^{2})
$$

and the potential is

$$
V = m_{2}g(h - lsin(\alpha)).
$$

The Lagrangian becomes

$$
L = \frac{1}{2}m_{1}\dot{x}^{2} + \frac{1}{2}m_{2}(2\dot{x}\dot{l}cos(\alpha) + \dot{l}^{2} + \dot{x}^{2}) - m_{2}g(h - lsin(\alpha))
$$


\subsection*{2b.}
\noindent
The potential energy of the system is actually 

$$
V = kx^{2} + 2kL(L - \sqrt{x^{2} + y^{2}})
$$
\noindent
since there is a need for 2 potential terms that may have different lengths at different times.

\noindent
The final Lagrangian becomes

$$
L = \frac{1}{2}m(\dot{x}^{2} + \dot{y}^{2}) - kx^{2} + 2kL(L - \sqrt{x^{2} + y^{2}})
$$

\subsection*{2c.}
\noindent
Let $\theta = \omega t$ and let 

\begin{align}
x = Rcos(\theta) + lsin(\phi) \\
y = Rsin(\theta) - lsin(\phi)
\end{align}
\noindent
We have that 

$$
T = \frac{1}{2}m(\dot{x}^{2} + \dot{y}^{2}) 
$$
$$
=\frac{1}{2}m(-R\omega sin(\theta) + l\dot{\phi}cos(\phi))^{2} + \frac{1}{2}m(R\omega cos(\theta) + l\dot{\phi}sin(\phi))^{2}.
$$
\noindent
Which will reduce down to 

$$
\frac{1}{2}m(R^{2}\omega^{2} + 2R\omega l \dot{\phi}sin(\phi - \theta) + l^{2}\dot{\phi}^{2}).
$$
\noindent
The potential energy for the system is 

$$
V = mg(Rsin(\theta) - lcos(\phi)).
$$
\noindent
Thus, the Lagrangian is 

$$
L = \frac{1}{2}m(R^{2}\omega^{2} + 2R\omega l \dot{\phi}sin(\phi - \theta) + l^{2}\dot{\phi}^{2}) - mg(Rsin(\theta) - lcos(\phi)).
$$

\subsection*{2d.}
\noindent
If $x = x - l$ then l would go to zero since the time derivative of l with respect to x is 0. I need to watch my derivatives closer and make better sense of them to avoid this problem. 

\section*{Problem 3}

The system constrains are based on the length of the rod and the angle the masses depend on. Therefore we have that the virtual work is 

\begin{align}
x &= lcos(\theta) & x\delta = lcos(\theta)\delta\theta \\
y &= lsin(\theta) & y\delta = lsin(\theta) \delta\theta
\end{align}

For virtual work we have that 

$$
F_{i}\frac{\partial x_{i}}{\partial q_{\alpha}}.
$$

Which will become

$$
mg\delta y = -F\delta x \rightarrow mg\delta y + F\delta x = 0.
$$

Finally, let's solve for F,

$$
F = mg\frac{lcos(\theta)\delta \theta}{lsin \theta \delta \theta} \rightarrow F = mgcot(\theta).
$$


\section*{Problem 4}
\noindent
I tend to have problems with my signs when writing out the Lagrangian. When I right it in the form of 

$$
\frac{d}{dt} \frac{\partial L}{\partial \dot{q}} = \frac{\partial L}{\partial q}.
$$
\noindent
I do tend to have more success when writing it in the form of 

$$
\frac{d}{dt} \frac{\partial L}{\partial \dot{q}} - \frac{\partial L}{\partial q} = 0.
$$
\noindent
This allows me to control most of the problems I may have with the signs. In this case on the exam I must have skipped over it. Making sense of the final equation of motion would also help me determine if it were correct or not. A $-m_{2}g$ makes more sense than a $+m_{2}g$ for this system, since $m_{2} > m_{1}$ the mass $m_{2}$ will be moving downwards.

% --------------------------------------------------------------
%                           End Document.
% --------------------------------------------------------------
 
\end{document}