% --------------------------------------------------------------
%                           Set Up
% --------------------------------------------------------------
 
\documentclass[12pt]{article}
 
\usepackage[margin=1in]{geometry} 
\usepackage{amsmath,amsthm,amssymb}
 
\newcommand{\N}{\mathbb{N}}
\newcommand{\Z}{\mathbb{Z}}
 
\newenvironment{theorem}[2][Theorem]{\begin{trivlist}
\item[\hskip \labelsep {\bfseries #1}\hskip \labelsep {\bfseries #2.}]}{\end{trivlist}}
\newenvironment{lemma}[2][Lemma]{\begin{trivlist}
\item[\hskip \labelsep {\bfseries #1}\hskip \labelsep {\bfseries #2.}]}{\end{trivlist}}
\newenvironment{exercise}[2][Exercise]{\begin{trivlist}
\item[\hskip \labelsep {\bfseries #1}\hskip \labelsep {\bfseries #2.}]}{\end{trivlist}}
\newenvironment{problem}[2][Problem]{\begin{trivlist}
\item[\hskip \labelsep {\bfseries #1}\hskip \labelsep {\bfseries #2.}]}{\end{trivlist}}
\newenvironment{question}[2][Question]{\begin{trivlist}
\item[\hskip \labelsep {\bfseries #1}\hskip \labelsep {\bfseries #2.}]}{\end{trivlist}}
\newenvironment{corollary}[2][Corollary]{\begin{trivlist}
\item[\hskip \labelsep {\bfseries #1}\hskip \labelsep {\bfseries #2.}]}{\end{trivlist}}

\newenvironment{solution}{\begin{proof}[Solution]}{\end{proof}}
 
\begin{document}
 
% --------------------------------------------------------------
%                         Start here
% --------------------------------------------------------------
 
\title{Quiz 6}
\author{Timothy Holmes\\ %replace with your name
Classical Mechanics}

\maketitle

\section*{1.}

I found it interesting that when

$$
\frac{dH}{dt} = 0 
$$

\noindent
that H is constant. What this means for the Hamiltonian is that H is E. Moreover, shows us that energy is conserved in the same way the Lagrangian showed us that momentum was conserved. 

\section*{2.}

The equation tells us that we can immediately identify it as energy. The equation is not an explicit function of time because the hamiltonian can be identified with energy. This implies energy of conservation.


\section*{3.}

If we have a point in phase space then only one path can be traced through that point. This is because if another path were to cross that point, it has to be a unique solution to the Hamilton's equation. However, that is not possible so it has to be the same path. 


% --------------------------------------------------------------
%                           End Document.
% --------------------------------------------------------------
 
\end{document}