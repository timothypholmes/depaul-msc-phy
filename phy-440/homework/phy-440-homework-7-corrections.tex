% --------------------------------------------------------------
%                           Set Up
% --------------------------------------------------------------
 
\documentclass[12pt]{article}
 
\usepackage[margin=1in]{geometry} 
\usepackage{amsmath,amsthm,amssymb}
\usepackage{listings}
\usepackage{xcolor}
\usepackage{graphicx}
\usepackage{subcaption}
 
\definecolor{codegreen}{rgb}{0,0.6,0}
\definecolor{codegray}{rgb}{0.5,0.5,0.5}
\definecolor{codepurple}{rgb}{0.58,0,0.82}
\definecolor{backcolour}{rgb}{0.95,0.95,0.92}
\definecolor{deepblue}{rgb}{0,0,0.5}
\definecolor{deepred}{rgb}{0.6,0,0}
\definecolor{deepgreen}{rgb}{0,0.5,0}
 
\lstdefinestyle{mystyle}{
    backgroundcolor=\color{backcolour},   
    commentstyle=\color{codegreen},
    keywordstyle=\color{deepred},
    numberstyle=\tiny\color{codegray},
    stringstyle=\color{deepblue},
    basicstyle=\ttfamily\footnotesize,
    breakatwhitespace=false,         
    breaklines=true,                 
    captionpos=b,                    
    keepspaces=true,                 
    numbers=left,                    
    numbersep=5pt,                  
    showspaces=false,                
    showstringspaces=false,
    showtabs=false,                  
    tabsize=2
}
 
\lstset{style=mystyle}
 
\newcommand{\N}{\mathbb{N}}
\newcommand{\Z}{\mathbb{Z}}
 
\newenvironment{theorem}[2][Theorem]{\begin{trivlist}
\item[\hskip \labelsep {\bfseries #1}\hskip \labelsep {\bfseries #2.}]}{\end{trivlist}}
\newenvironment{lemma}[2][Lemma]{\begin{trivlist}
\item[\hskip \labelsep {\bfseries #1}\hskip \labelsep {\bfseries #2.}]}{\end{trivlist}}
\newenvironment{exercise}[2][Exercise]{\begin{trivlist}
\item[\hskip \labelsep {\bfseries #1}\hskip \labelsep {\bfseries #2.}]}{\end{trivlist}}
\newenvironment{problem}[2][Problem]{\begin{trivlist}
\item[\hskip \labelsep {\bfseries #1}\hskip \labelsep {\bfseries #2.}]}{\end{trivlist}}
\newenvironment{question}[2][Question]{\begin{trivlist}
\item[\hskip \labelsep {\bfseries #1}\hskip \labelsep {\bfseries #2.}]}{\end{trivlist}}
\newenvironment{corollary}[2][Corollary]{\begin{trivlist}
\item[\hskip \labelsep {\bfseries #1}\hskip \labelsep {\bfseries #2.}]}{\end{trivlist}}

\newenvironment{solution}{\begin{proof}[Solution]}{\end{proof}}

\setlength\parindent{0pt}
 
\begin{document}
 
% -------------------------------------------------------------- 
%                         Start here
% --------------------------------------------------------------
 
\title{Homework 7 Corrections}
\author{Timothy Holmes\\ %replace with your name
PHY 440 Classical Mechanics}

\maketitle

\section*{Problem 1}

\subsection*{a.}

$Y = -p/mg$, need to track my signs better. Especially when taking derivatives.

\subsection*{b.}

Both the 

$$
\dot{Y} = \frac{\partial K}{\partial P}
$$

and

$$
\dot{P} = -\frac{\partial K}{\partial \dot{Y}}
$$

need to be taken.

$$
\dot{Y} = 1, \dot{P} = 0.
$$

Integrating these equations for Y(t) and P(t) we get,

$$
Y(t) = t + C, P(t) = E.
$$

\subsection*{c.}

Going back to part a and combining with part b we have 

$$
Y = -\frac{p}{mg} \rightarrow -mgY(t) \rightarrow -mg(t + C)
$$

Initial conditions show that t = 0 so

$$
p(0) = -v_{0}/g
$$

and 

$$
Y(0) = -v_{0}^ = h
$$

Finally,

$$
E = \frac{m v_{0}}{2} + mgh
$$


\subsection*{d.}

Lastly we need to find another canonical transformation given the generating function. Where the generating function is 

$$
F_{4} = \frac{p^{3}}{6m^{2}g} - p\frac{R}{mg} - Rt
$$

We find that 

$$
H = \frac{p^{2}}{2m} + mgy = R
$$

and 

$$
Z = -\frac{p}{mg} - t.
$$

know $H$ and $\partial F_4/ \partial t$ 

$$
K = 0
$$

$Z(t)$ and $R(t)$ can be found by 

$$
\dot{Z} = 0 \rightarrow Z(t) = A
$$

$$
\dot{R} = 0 \rightarrow = B.
$$

The initial conditions where t = 0 yield

$$
R(0) = \frac{mv_{0}^{2}}{2} + mgh
$$

and 

$$
Z(0) = -\frac{v_{0}}{g}.
$$

% --------------------------------------------------------------
%                           End Document.
% --------------------------------------------------------------
 
\end{document}

