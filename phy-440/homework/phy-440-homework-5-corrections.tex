% --------------------------------------------------------------
%                           Set Up
% --------------------------------------------------------------
 
\documentclass[12pt]{article}
 
\usepackage[margin=1in]{geometry} 
\usepackage{amsmath,amsthm,amssymb}
\usepackage{listings}
\usepackage{xcolor}
\usepackage{graphicx}
\usepackage{subcaption}
 
\definecolor{codegreen}{rgb}{0,0.6,0}
\definecolor{codegray}{rgb}{0.5,0.5,0.5}
\definecolor{codepurple}{rgb}{0.58,0,0.82}
\definecolor{backcolour}{rgb}{0.95,0.95,0.92}
\definecolor{deepblue}{rgb}{0,0,0.5}
\definecolor{deepred}{rgb}{0.6,0,0}
\definecolor{deepgreen}{rgb}{0,0.5,0}
 
\lstdefinestyle{mystyle}{
    backgroundcolor=\color{backcolour},   
    commentstyle=\color{codegreen},
    keywordstyle=\color{deepred},
    numberstyle=\tiny\color{codegray},
    stringstyle=\color{deepblue},
    basicstyle=\ttfamily\footnotesize,
    breakatwhitespace=false,         
    breaklines=true,                 
    captionpos=b,                    
    keepspaces=true,                 
    numbers=left,                    
    numbersep=5pt,                  
    showspaces=false,                
    showstringspaces=false,
    showtabs=false,                  
    tabsize=2
}
 
\lstset{style=mystyle}
 
\newcommand{\N}{\mathbb{N}}
\newcommand{\Z}{\mathbb{Z}}
 
\newenvironment{theorem}[2][Theorem]{\begin{trivlist}
\item[\hskip \labelsep {\bfseries #1}\hskip \labelsep {\bfseries #2.}]}{\end{trivlist}}
\newenvironment{lemma}[2][Lemma]{\begin{trivlist}
\item[\hskip \labelsep {\bfseries #1}\hskip \labelsep {\bfseries #2.}]}{\end{trivlist}}
\newenvironment{exercise}[2][Exercise]{\begin{trivlist}
\item[\hskip \labelsep {\bfseries #1}\hskip \labelsep {\bfseries #2.}]}{\end{trivlist}}
\newenvironment{problem}[2][Problem]{\begin{trivlist}
\item[\hskip \labelsep {\bfseries #1}\hskip \labelsep {\bfseries #2.}]}{\end{trivlist}}
\newenvironment{question}[2][Question]{\begin{trivlist}
\item[\hskip \labelsep {\bfseries #1}\hskip \labelsep {\bfseries #2.}]}{\end{trivlist}}
\newenvironment{corollary}[2][Corollary]{\begin{trivlist}
\item[\hskip \labelsep {\bfseries #1}\hskip \labelsep {\bfseries #2.}]}{\end{trivlist}}

\newenvironment{solution}{\begin{proof}[Solution]}{\end{proof}}

\setlength\parindent{0pt}
 
\begin{document}
 
% -------------------------------------------------------------- 
%                         Start here
% --------------------------------------------------------------
 
\title{Homework 5 Corrections}
\author{Timothy Holmes\\ %replace with your name
PHY 440 Classical Mechanics}

\maketitle

\section*{Problem 1}

I must have accidentally deleted what I had when adding a sentence below. I should check over my homework in more detail before submitting it. The equation of motion for problem 1 for $\theta$ is 

$$
\frac{d}{dt}(2m_{1}r^{2}\dot{\theta}) = 0 
$$

and is constant.


\section*{Problem 2}

I drop one of the $\lambda$ terms for my $x_{1}$ term and my $f_{3}$ constraint. Furthermore, I stopped short of solving for all of the $\Lambda_{i}$ values. I found that my $\Lambda_{1}$ values is 0. I then should of found that 

$$
\Lambda_{2} = \frac{g\sqrt{M^{2} - m^{2}}{2b}}{2b}
$$

and 

$$
\Lambda_{3} = g\sqrt{M^{2} - m^{2}}.
$$

Finally, after finding the $\Lambda$ values I could then solve for the forces.
Where

$$
Q_{x_{2}} = 0 
$$

$$
Q_{y_{1}} = mg
$$

$$
Q_{y_{2}} = Mg
$$
% --------------------------------------------------------------
%                           End Document.
% --------------------------------------------------------------
 
\end{document}

