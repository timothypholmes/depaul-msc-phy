\documentclass[a4paper]{article}

\usepackage[english]{babel}
\usepackage[utf8]{inputenc}
\usepackage{amsmath}
\usepackage{graphicx}
\usepackage{epsfig}
\usepackage[colorinlistoftodos]{todonotes}
\usepackage[hidelinks]{hyperref}
\usepackage[margin=1.3in]{geometry}

\title{Alternative Gravitational Theory}

\author{Timothy Holmes \\ 
        DePaul University}

\date{\today}

\newcommand{\gfigure}[4]
{
	\begin{figure}
      \begin{center}
          \includegraphics[#3]{#4}
          \caption{#1}
          \label{Figure:#2}
      \end{center}
    \end{figure}
}
\newcommand{\figref}[1]{Figure~\ref{Figure:#1}}

\begin{document}
\maketitle

\begin{abstract}
The theory of gravity has evolved over time. Starting with Newton's laws and his theory of gravity, the dynamics of motion can be described very well but at the cost of some accuracy. Over two centuries later, Einstein reveled his theory of General Relativity (GR). GR is the best description of gravity as of today, even being a century old. New observations and phenomena such as dark matter require a new description of gravity that GR can not describe. Currently, under the family of $f(R)$, the Chameleon Theory has promising results, theorizing a fifth force that helps explain gravity on quantum scales. An experiment in space has detected the fifth force. A simulation of this theory also proves that the creation of galaxies are possible using the Chameleon mechanism, which uses $f(R)$ gravity, a modified GR Theory.
\end{abstract}

\section{Introduction}

As the exploration of the Universe continues technology continues to improve. This in return helps further the evidence to support theories about many questions of the Universe. Currently, Cosmology is in the "Golden Era" of research much like its counterparts in the 1920 for the field of physics. One can say it is a very highly active field of physics today. However, as of today many new questions are arising and many problems are being brought up from these new observation methods. Many of these problems rely on our understanding of gravity, but gravity has always been a challenging topic. The first accurate theory of gravity was from Newton in his Principa Mathematica (1687) \cite{J__2011}. Newton was great at describing the motion of matter in a way no one else could at the time. Formerly, over two centuries later Einstein came out with his paper on special and General Relativity. General Relativity is now over a century old and is still has the best description of gravity today. 

 If General Relativity described gravity without any flaws, there would be no need for alternative gravitational theories. Nonetheless, there are some observations in the cosmos that can not be described by Einsteins General Relativity, recently speaking. Many of these alternative theories have been created for thoughts of there being more forces than presently know. But to say with certainty the most popular theories are one created to describe dark matter and dark energy. The dark sector, is what dark matter and dark energy is commonly referred to\cite{Nojiri_2017}.  Within itself dark matter and dark energy are monumental in the study of cosmology and large enough to be part of a different topic. A slight overview, however, is that dark matter is currently the best reasoning to explain the unexpected results of rotation in a galaxy and the only evidence of it comes from light distorted by it from the effect of gravitational lensing. Some other theories that are important theories that try and explain gravity in higher dimensions. The motivation for this is to try and unify different fields. Other popular theories are motivated around phenomenons like the big bang and are modeled to match different scenarios to help explain today's world. Another paramount
alternative theory  of gravity is quantum gravity. While general relativity explains gravity well for the classical world, even Newton's gravity theory does this to a degree. Neither theories explain gravity in the quantum world. This is problematic and renders the modern day inability to find a solution to the problem, leaving this a colossal puzzle in modern physics. 

Between now and the publication of General Relativity, there has been an on going list of different gravitational theories. Many can be classified into some smaller sub category as many are similar in nature. The motivation from each new alternative gravitational theory varies. New theories are introduced to solve problems based on other theories while other gravitational theories are constructed to explain phenomenons in the universe. Whatever the cause, the fact on why there are an abundant amount of alternative theories may be stem from the lack of fully explaining the universe. Many theories may match each other and explain phenomenons well, while other theories match another phenomenon well but the two theories might not match the same gravitational theory well. This is the purpose of these separate theories. 

\subsection{Newtonian Mechanics}

In classical mechanics Newton's ideas are used frequently to describe motion. Newton's three fundamental laws are the underlying structure how to describe motion and the simplest way to describe motion. Therefore, remembering that;  a particle in motion will stay in motion unless acted on by an external force, $\vec{F} = m\vec{a}$, and for every action there is an equal and opposite reaction. These three fundamental laws are typically enough to describe the motion of an object for most situations. These laws are simplistic and can explain the motion of many objects. One of Newton's most useful equation used in cosmology is how the motion of planets is describes given as 

\equation{
F = -\frac{Gm_{1}m_{2}}{r^{2}}
}

where $G$ is the gravitational constant, $m_{1}$ is the mass of the first object, $m_{2}$ is the mass of the second object, and $r$ is the distance between both objects. This is the main idea of gravity for Newton, simply put gravity is a force that causes massive bodies to accelerate toward each other. Newton's idea of the universe was that it is static with Euclidean geometry. 

Although Newton's theory of gravity is old it does not mean it is outdated. Moreover, in comparison, even though general relativity is old it is still not outdated. There are appropriate times to use each theory and times when there are not. In the case for Newton's gravitational theory it turns out that it is profound at explaining the structure of the sun \cite{hartle_2003}.

\section{General Relativity}
General Relativity within itself is a very dense topic and can be a time consuming subject on its own. However, it is important to understand since it is the most accurate theory of gravity currently. More importantly, alternative gravitational theories are mostly alternatives to general relativity. Prior to Einstein, Newton had the best description of gravity. Moreover, Prior to Einstein's general theory of general relativity came his theory of special relativity. The theory excluded gravity and also considered the universe to have Euclidean geometry. According to different observers reference frames, both observers would agree on everything except for the time measurement. This leads to the idea of a four-dimensional spacetime where space and time is connected. However, as previously stated, special relativity isn't a gravitational theory. The inconsistency between Einstein and Newton lead Einstein to his relativistic theory of gravity.

The idea of a curved spacetime no longer allows Euclidean geometry to hold. Since spacetime is a four-dimensional union of space and time, gravity must be thought as the geometry of both space and time. Essentially what this means, instead of treating gravity as a force like Newton did, Einstein treated the "gravitational force" as geometry. The transition from space to spacetime includes the use of geodesics. A straight line on a plane from point A to point B is an example of the shortest distance in Euclidian Geometry. When it comes to curved spacetime, the shortest straight line between two points A and B has to be through curved spacetime, this is called a geodesic. However, it is not to think of a geodesic as a line bust as a curve with proper time.



\section{Alternative Theories}

\subsection{$f(R)$ Modified Gravity}

The theory of $f(R)$ gravity can be considered as a family of theories, such as the chameleon Theory that builds its ideas off of $f(R)$ gravity. The basic idea of this theory starts with the Einstein-Hilbert action given as 

$$
S = \frac{1}{2k}\int d^{4}x\sqrt{-g}
$$

where $k = 8\pi G/c^{4}$, $g$ is the determinate of a tensor, and $d^{4}x$ is the volume of a four dimensional object. Typically, the four dimensional object is the wrote as $L$ since it is the Lagrangian, for simplicity. This is known as the Lagrangian scalar. This Lagrangian scalar often substitutes the Ricci scalar in. In general, the Ricci scalar is just a tensor that helps describe the geometry. The Einstein-Hilbert action then tend to be wrote as

$$
S = \frac{1}{2k}\int d^{4}x\sqrt{-g}R
$$

where R is the Ricci scalar. The idea behind this is that this equation helps to explain the dynamics of GR and one solved for will describe the energy-momentum tensor. This can also be know as the stress tensor and it helps to explain the relationship between energy and momentum of matter.

Much like the Friedmann equation that helps to explain the universe, $f(R)$ gravity helps to explain gravity in different situations. This in a sense can be thought of as a toy, toy gravity, that by manipulating changes the way gravity is explained. Many of gravitational experiments are simulated using the power of super computers. Essentially, physicist take the idea of gravity from GR and manipulate it such that it works for the scenarios they are trying study. This could include the early universe with the goal of explaining the Big Bang. By manipulating what Einstein found in GR, the new manipulated theory can explain gravity of different scales. Again in the early universe it could explain when the universe did not collapse on itself and cause the Big Crunch. Due to its scaling capabilities it could explain gravity on quantum scales and explain why the universe didn't expand such that it results in the Big Freeze. 

\subsubsection{Chameleon Theory}

The Chameleon fifth force theory is a leading theory that supports the idea of dark matter and dark energy in the universe. One of many unexplained problems in cosmology is why the universe may be expanding. One theory that is trying to solve this problem is the Chameleon fifth force problem. Recently, there has been a lot of speculation and support towards this gravitational theory. 

The thought process goes back to dark matter and dark energy and explains that if this matter is expanding the universe, then, the observation of this should come with an extra force. This "force", however, is primarily undetected and unobserved. Since dark energy is abundant in the universe, there should also be a lot of what is called chameleon particles. These particles are in a sense massive and have the capabilities to change mass, making them hard to detect. Heavy particles only have affects on small scales and light particles have affects on large scales. Therefore,  the chameleon particle can affect the universe in such a way that it has affects on the expansion of the universe when it is a small particle, but then has no measured force when it is a small particle. This is when quantum mechanics becomes involved, this is due to the detection on small scales. 


\subsubsection{Recent activity with the Chameleon Theory}

There has been a lot of recent activity in this field and a number of on going experiments that are trying to detect this force. There are experiments in space that are conducted with satellites, here on earth, and through the process of computer simulations.

One relatively recent and successful experiment was one in space, testing the fifth force in a gravity space experiment. The mechanism for the chameleon theory is as follow

$$
S = \int dx^{4}\sqrt{-g}\Big[\frac{M^{2}_{Pl}}{2}R - \frac{1}{2}\partial^u \partial_{\mu}\phi - V(\phi)\Big] - \int d^{4}xL_{m}(\Tilde{g}_{\mu v}, \psi, ...)
$$
where $\psi$ is the chameleon field, $V$ is the potential, $M_{Pl}$ is the reduced Plank mass, $R$ is the Ricci scalar, $g_{\mu v}$ is the Einstein frame metric, and $L$ is the Lagrangian matter \cite{Pernot_Borr_s_2019}. This resembles the Einstein-Hilbert action as shown previously. The additive terms act in the equation act as a scaling factor. This experiment is conducted in space to try to ignore the effects that might be caused here on Earth. Due to the measurements being on the quantum scale, accuracy is important to determine proper results. A single atom is passed to try and determine if there is an external force affecting it, this would be the fifth force.

\subsubsection{Simulation of $f(R)$ gravity}

A simulation of $f(R)$ gravity was recently published. This simulation also was a chameleon-typed theory of modified gravity. The paper used the following Einstein-Hilbert action

$$
S = \int d^{4}x\sqrt{-g}\Big[\frac{R + f(R)}{16\pi G} + L \Big].
$$


Similar to the last Einstein-Hilbert action this one includes G as the gravitational constant, g is the determinant of the metric $g_{\mu v}$ and $L$ is the Lagrangian density. This equation adds a scalar degree of freedom which enhances the gravitational force for a low-density environment \cite{Arnold_2019}. GR was unable to explain the movement of stars around galaxies, which is what is seen in Kepler's predictions versus what is observed. Gr was also unable to explain the expansion of the universe as previously explained. The key is that dark matter and dark energy is to blame. Therefore, adding in these extra terms to the Einstein-Hilbert action modify what GR is trying to predict. This free function, as seen in the last equation, varies the Einstein-Hilbert action. Now, there is this toy gravity that can be manipulated to try and discover the fix to these problems. In doing so, everything has to be resolved for and tested to prove that the new modified theory works. Therefore, prior test that GR had to use need to be re-observed using the new modified theory. 

\section{Conclusion}

To conclude, Newton's laws were able to explain the motion of objects well, so much so that they are still considered well enough for most measurements today. Newton's gravitational theory is same in nature and is good enough for most today. However, ideas get outdated and surpassed sue to better ideas or the discovery of them being inaccurate. In this case, Einstein came along with his theory of GR, a successful and accurate theory to gravity. This theory of GR is perfect for most cases and accurate, accurate enough to explain the motion of satellites that keep GPS working. Which in fact needs to be precise. The downfall is that GR only explains gravity in the classical world and there is currently no working theory that explains gravity in the quantum world. Further more, there tends to be new observations in the universe, dark matter and dark energy, that are hard to be explained using GR. The alternative theory that is $f(R)$ helps modify GR where it is able to fix some of these issues. A fifth force has be detected in the space experiment. With the theory of the fifth force, it could help to explain why the universe is expanding the way it is and the early universe. 

\nocite{*}
\bibliographystyle{IEEEannot}
\bibliography{thesis}
\end{document}